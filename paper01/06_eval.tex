% !TeX root = 00P1_cae_ontology.tex

\section{Evaluation via Competency Questions}
\label{sec:evaluation}

This section evaluates the adequacy of the CAE ontology using explicit
competency questions grounded in real civic data.
The goal is not exhaustive validation, but to demonstrate that the
ontological distinctions introduced by CAE enable representations that
are difficult or ambiguous under common alternative models.

\subsection{Explicit Competency Questions}

The following competency questions guided the design of CAE and are used
here to assess its expressive adequacy:

\begin{itemize}[nosep]
      \item \textbf{CQ1:} Can the ontology distinguish enduring sources of authority
            from the time-indexed events that realize or execute that authority?

      \item \textbf{CQ2:} Can a single civic entity participate simultaneously in
            multiple accountability roles (e.g., regulator, operator, recipient)
            without reclassification or type mutation?

      \item \textbf{CQ3:} Can outcomes and public value be represented as
            observable entities without asserting causal claims or embedding
            evaluative judgments in the ontology?

      \item \textbf{CQ4:} Can accountability chains be traced across jurisdictions
            and over time without redefining entity identity?

      \item \textbf{CQ5:} Can delegation of authority be represented explicitly
            through instruments, rather than implicitly through role-based entity types?
\end{itemize}

These questions are answered through a worked mapping of a real public
grant record and a comparative discussion with an established provenance model.

\subsection{Dataset Mapping: USAspending Grant Example}

To illustrate these competency questions, we consider a representative
grant record published via USAspending.gov.
The example is intentionally minimal and descriptive; it is not intended
as a comprehensive schema mapping.

A typical grant record involves the following elements:

\begin{itemize}[nosep]
      \item \textbf{Actors (A):} a federal awarding agency and a recipient organization
      \item \textbf{Instruments (I):} an enduring grant program and a specific grant award
      \item \textbf{Events (E):} obligation and payment events occurring at specific times
      \item \textbf{Jurisdictions (J):} federal and state jurisdictions governing applicability
      \item \textbf{Observations (O):} optional outcome indicators (e.g., program-level performance metrics)
\end{itemize}

In CAE, the grant program is modeled as an enduring \emph{Instrument}
that grounds authority and eligibility conditions.
The grant award and subsequent payments are modeled as \emph{Events}
occurring under that instrument.
The recipient organization remains a single \emph{Actor} throughout,
even as it occupies the roles of applicant, awardee, and reporting entity.

This representation directly addresses CQ1, CQ2, and CQ4 by separating
authority from execution, preserving entity identity, and allowing
longitudinal analysis across jurisdictions.

\subsection{Comparative Limitation of Provenance-Oriented Models}

In widely used provenance ontologies such as PROV-O,
both grant programs and grant awards are typically represented as
\texttt{prov:Entity}, while payments are represented as
\texttt{prov:Activity}.
While sufficient for recording provenance, this structure does not
enforce a distinction between enduring normative constructs and
time-indexed realizations.

As a result, expressing the difference between a standing source of
authority (e.g., a grant program) and the events that execute or fulfill
that authority requires additional modeling conventions external to the
ontology.
Similarly, role distinctions (e.g., regulator versus operator) must be
handled informally or through application-specific typing.

\subsection{CAE Representation of the Same Scenario}

CAE addresses these limitations through its disjoint partition of entity kinds.
In the same grant scenario:

\begin{itemize}[nosep]
      \item The grant program is modeled as an \emph{Instrument} (enduring authority).
      \item The award and payments are modeled as \emph{Events} (time-indexed occurrences).
      \item The recipient organization remains a single \emph{Actor}, with roles
            emerging from relationship patterns rather than reclassification.
\end{itemize}

No subclassing, role entities, or ad hoc conventions are required.
Authority, execution, and observation are separated structurally rather
than procedurally.

\subsection{Discussion of Results}

These examples demonstrate that CAE's disjoint partition enables
ontological distinctions that are not enforced in
provenance-oriented or transaction-centric models.
In particular, CAE supports explicit representation of authority,
delegation, execution, and outcomes while remaining neutral with respect
to causal interpretation or policy evaluation.

The evaluation indicates that CAE satisfies the stated competency
questions and provides a stable ontological foundation for downstream
exchange semantics and explanatory frameworks, without requiring their
inclusion in the present paper.

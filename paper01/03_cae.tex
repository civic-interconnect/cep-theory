% !TeX root = 00P1_cae_ontology.tex

\section{Ontological Partition of Accountable Entities}
\label{sec:ontology}

This section defines the core ontological commitment of the Civic Accountable
Entities (CAE) framework.
CAE introduces a strict partition of accountable entities into six disjoint kinds:
Actors (A), Sites/Assets (S), Instruments (I),
Events (E), Jurisdictions (J), and Observations (O).
Each entity instantiated within CAE is assigned exactly one kind.
Entity kinds are invariant over time:
entities do not change kind, and apparent role changes are represented through
relationships rather than reclassification.

The partition is designed to support formal semantics of exchange, authority,
and evidence while preventing ontological overlap.
Inclusion of an entity in CAE is driven by participation in obligations, authority relationships, or
accountability-bearing exchanges, rather than by descriptive completeness or
sectoral classification.

\subsection{Actors (A)}
\label{subsec:actors}

Actors are entities capable of bearing rights, obligations, or responsibilities
within civic systems.
An Actor may initiate, receive, authorize, or be held
accountable for actions governed by Instruments and manifested through Events.
Actors are the only entity kind that may serve as parties to obligations.

Examples of Actors include governments, public agencies, private businesses,
nonprofit organizations, universities, research institutes, and think tanks.
Public or private status, sector, mission, and organizational role are treated
as attributes or relationships, not as entity kinds.

An Actor is introduced into CAE only when it participates in an accountability
relationship, such as receiving funds, issuing authority, operating regulated
Sites, or being subject to reporting or enforcement.
CAE does not attempt to enumerate organizations exhaustively;
rather, it models Actors insofar as they
are implicated in civic exchanges or obligations.

\subsection{Sites and Assets (S)}
\label{subsec:sites}

Sites and Assets are physical or operational entities that are acted upon but do
not themselves bear obligations.
They provide the spatial, infrastructural, or
material substrate upon which civic activity occurs.
Sites and Assets may be owned, operated, regulated, inspected, or measured,
but they are not parties to Instruments.

Examples include facilities, buildings, campuses, power plants, laboratories,
stores, transportation infrastructure, and other physical installations.
Geographic location is an intrinsic property of Sites and Assets and provides a
natural point of attachment to Jurisdictions.

Treating Sites and Assets as a distinct entity kind ensures a clear separation
between accountable actors and the physical or operational entities through
which obligations are exercised or impacts are realized.

\subsection{Instruments (I)}
\label{subsec:instruments}

Instruments are enduring constructs that create, modify, delegate, or constrain
rights, obligations, or authority.
Instruments mediate relationships between Actors and govern the conditions under which Events may occur.
Instruments are not time-indexed occurrences but persistent normative or programmatic entities.

CAE distinguishes conceptual subclasses of Instruments without introducing
additional entity kinds.
These include normative instruments (such as statutes
and laws), regulatory instruments (such as rules and regulations), and
programmatic instruments (such as contracts, grants, permits, licenses, and
formal programs).
All such instruments share the defining property of grounding accountability relationships.

Instruments are included in CAE only when they give rise to concrete Events or Observations.
This design avoids exhaustive modeling of legal or administrative
texts while preserving the causal and accountability structure necessary for
exchange and evidence semantics.

\subsection{Events (E)}
\label{subsec:events}

Events are time-indexed occurrences asserted under the authority of Instruments.
An Event records that something happened at a particular time and place and may
involve one or more Actors, Sites, or Jurisdictions.
Events constitute the primary evidence of activity within civic systems.

Examples include payments, inspections, filings, emissions submissions,
violations, audits, and enforcement actions.
Events are not enduring objects;
their identity is inseparable from their temporal occurrence and provenance.

By separating Events from Instruments, CAE distinguishes between the existence
of obligations and their execution or violation.
This separation is essential for representing compliance, performance, and accountability over time.

\subsection{Jurisdictions (J)}
\label{subsec:jurisdictions}

Jurisdictions are entities that scope authority, applicability, and governance.
They define where Instruments apply, where Events may occur, and where Observations are interpreted.
Jurisdictions are not Actors; they do not initiate actions or bear obligations,
but they structure accountability relationships.

Examples include nations, states, provinces, municipalities, regulatory regions,
air basins, and watersheds.
Jurisdictions may be nested or overlapping, and such
structure is represented explicitly through relationships.

Treating Jurisdictions as a distinct entity kind allows CAE to model legal,
regulatory, and environmental scope without conflating authority with agency or
action.

\subsection{Observations (O)}
\label{subsec:observations}

Observations are measurements or indicators describing the state, performance,
or outcomes associated with Actors, Sites, Instruments, Events, or Jurisdictions.
Observations do not create obligations and do not represent actions; they assert
measured or derived facts with associated provenance.

Examples include health outcomes, coverage rates, emissions intensity, food
access indicators, educational attainment measures, and other longitudinal or
comparative metrics.
Observations may be aggregated, statistical, or model-based
and are typically associated with populations or regions through attributes and
relationships.

Introducing Observations as a first-class entity kind enables CAE to represent
long-term public value, outcomes, and impacts without collapsing measurement into
events or instruments.
This separation supports comparative analysis across time
and jurisdictions while remaining neutral with respect to causal interpretation.
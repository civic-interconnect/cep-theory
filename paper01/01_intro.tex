% !TeX root = 00P1_cae_ontology.tex

\section{Introduction and Motivation}
\label{sec:introduction}

Civic systems are governed by complex interactions among laws, institutions,
infrastructure, financial flows, and measured outcomes.
Data describing these systems is typically fragmented across domains such as procurement, public
health, environmental regulation, education, and infrastructure.
While each domain is often supported by mature data systems, the structural relationships
that connect authority, obligation, action, and long-term outcomes are
rarely represented in a unified or
interoperable form~\cite{bowker2000sorting,edwards2011infrastructure}.

This fragmentation presents a fundamental obstacle to accountability and
longitudinal analysis.
Short-term metrics are frequently privileged over
long-term public value, and outcomes that accrue over decades—such as population
health, infrastructure resilience, or environmental quality—are difficult to
relate back to the legal, institutional, and financial decisions that shape
them~\cite{edwards2011infrastructure,kahn2002information}.
As a result, public investments with high upfront costs and diffuse
benefits are systematically undervalued, even when their historical impact is
well established.

Many existing approaches focus either on transactional data (such as payments
or contracts), legal texts (such as statutes and regulations), or outcome
measures (such as health or economic indicators).
Few provide a principled way to connect these elements without
collapsing distinct concepts into a single
layer or embedding interpretive assumptions directly into data models~\cite{bowker2000sorting}.

This paper introduces the Civic Accountable Entities (CAE) ontology as a
foundational response to this challenge.
CAE defines a formal ontology of entities that participate in
obligations, authority, and accountability within civic systems.
Rather than modeling domains or sectors directly, CAE identifies
a small set of disjoint entity kinds that are stable across time and context and
sufficient to represent the structural relationships underlying civic accountability.

The design of CAE emphasizes ontological clarity over descriptive completeness.
Entities are partitioned into disjoint kinds with explicit identity criteria;
roles, classifications, and sectoral labels are modeled as attributes or
relationships rather than as entity kinds.
This discipline prevents ontological overlap and supports formal reasoning
about obligations, authority, and evidence.

CAE draws on foundational work in formal ontology, particularly the
emphasis on rigorous categorization found in BFO~\cite{smith2015bfo} and
DOLCE~\cite{masolo2004wonderweb}, while making commitments tailored to
civic accountability: entities are included in CAE only insofar as they
participate in accountability-bearing relationships, and the ontology
is designed to remain stable as domains, policies, and tooling evolve.
Section~\ref{sec:related} discusses these relationships in detail.

CAE provides the ontological foundation for the Civic Exchange Protocol (CEP),
which models how entities exchange value and authority, and for Contextual
Evidence and Explanations (CEE), which attaches structured explanations to
civic decisions.
By separating what exists from how it moves and how decisions are explained,
CAE enables interoperable, auditable, and longitudinal analysis of public
systems while remaining neutral with respect to policy positions or causal claims.

The remainder of this paper is organized as follows:
Section~\ref{sec:related} situates CAE within related work in formal ontology;
Section~\ref{sec:principles} outlines the design principles and scope
of CAE;
Section~\ref{sec:ontology} defines the six disjoint entity kinds;
Section~\ref{sec:relationships} discusses relationships and structural constraints;
Section~\ref{sec:evaluation} evaluates CAE via competency questions;
Section~\ref{sec:laws} discusses laws, regulations, and accountability chains;
Section~\ref{sec:outcomes} discusses outcomes, observations, and public value;
Section~\ref{sec:discussion} includes discussion and future work;
and Section~\ref{sec:conclusion} provides a conclusion.

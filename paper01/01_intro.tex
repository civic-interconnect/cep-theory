% !TeX root = 00P1_cae_ontology.tex


\section{Introduction and Motivation}
\label{sec:introduction}

Civic systems are governed by complex interactions among laws, institutions,
infrastructure, financial flows, and measured outcomes.
Data describing these systems is typically fragmented across domains such as procurement, public
health, environmental regulation, education, and infrastructure.
While each domain is often supported by mature data systems, the structural relationships
that connect authority, obligation, action, and long-term outcomes are
rarely represented in a unified or interoperable form~\cite{bowker2000sorting,edwards2011infrastructure}.

This fragmentation presents a fundamental obstacle to accountability and
longitudinal analysis.
Short-term metrics are frequently privileged over
long-term public value, and outcomes that accrue over decades—such as population
health, infrastructure resilience, or environmental quality—are difficult to
relate back to the legal, institutional, and financial decisions that shape
them~\cite{edwards2011infrastructure,kahn2002information}.
As a result, public investments with high upfront costs and diffuse
benefits are systematically undervalued, even when their historical impact is
well established.

Many existing approaches focus either on transactional data (such as payments
or contracts), legal texts (such as statutes and regulations), or outcome
measures (such as health or economic indicators).
Few provide a principled way to connect these elements without
collapsing distinct concepts into a single
layer or embedding interpretive assumptions directly into data models~\cite{bowker2000sorting}.

This paper introduces the Civic Accountable Entities (CAE) ontology as a
foundational response to this challenge.
CAE defines a categorical ontology of entities that participate in
obligations, authority, and accountability within civic systems.
Rather than modeling domains or sectors directly, CAE identifies
a small set of disjoint entity kinds that are stable across time and context and
sufficient to ground formal semantics of exchange and evidence~ \cite{cep2025spec}.

CAE is intended to serve as the object-level foundation for subsequent category
semantics of the Civic Exchange Protocol (CEP) and Contextual Evidence and
Explanations (CEE).
By separating ontological commitment from exchange semantics
and evidentiary reasoning, CAE enables interoperable, auditable, and longitudinal
analysis while remaining neutral with respect to policy positions or causal
claims.
% !TeX root = 00P1_cae_ontology.tex

\section{Related Work}
\label{sec:related}

CAE is informed by foundational work in formal ontology while making
distinct commitments suited to civic accountability.
This section positions CAE relative to upper ontologies, ontological
methodology, and relevant domain standards.

\subsection{Upper Ontologies}

The Basic Formal Ontology (BFO)~\cite{smith2015bfo} provides
a realist upper ontology grounded in scientific practice.
BFO distinguishes continuants (entities that persist through time) from
occurrents (entities that unfold in time), a distinction reflected in
CAE's separation of enduring entities (Actors, Sites/Assets, Instruments,
Jurisdictions) from time-indexed entities (Events, Observations).
However, BFO adopts a realist stance in which entities are included by
virtue of mind-independent existence.
CAE takes a more operational approach: entities are included insofar as
they participate in accountability-bearing relationships, independent of
broader metaphysical commitments.

DOLCE (Descriptive Ontology for Linguistic and Cognitive
Engineering)~\cite{masolo2004wonderweb} adopts a descriptive orientation,
modeling categories as they are conceptualized rather than as they exist independently.
CAE shares this pragmatic stance but is substantially more constrained:
where DOLCE provides a general-purpose upper ontology with rich taxonomic
structure, CAE defines exactly six entity kinds with no subkind hierarchy
at the ontological level.
This minimalism is deliberate;
taxonomic and sectoral distinctions are deferred to vocabularies and attributes.

\subsection{Ontological Methodology}

The design of CAE reflects methodological principles from
OntoClean~\cite{guarino2002evaluating,guarino2009ontology}.
OntoClean emphasizes rigorous identity criteria and the distinction between
rigid properties (essential to an entity's identity) and anti-rigid
properties (borne contingently).
CAE's commitment to disjoint entity kinds reflects the OntoClean
requirement that taxonomic structures respect identity:
an entity's kind is rigid and invariant, while roles such as
regulator, funder, or recipient are modeled as anti-rigid
relational properties.

This treatment prevents the ontological confusion that arises when
roles are reified as entity types, a common source of inconsistency
in information systems~\cite{guarino2009ontology}.

\subsection{Provenance and Evidence}

The W3C PROV-O ontology~\cite{lebo2013prov} provides a widely adopted
model for representing provenance of entities, activities, and agents.
CAE's Events bear resemblance to PROV-O's Activity class, and
Observations relate to PROV-O's Entity as a record of state.
However, CAE makes distinctions not explicit in PROV-O.
Events in CAE are time-indexed occurrences asserted under the authority
of Instruments; they are not merely activities but accountability-bearing
assertions.
Observations are measurements or indicators with associated provenance,
distinct from the Events they may describe.

Instruments, as enduring normative constructs that create or constrain
obligations, have no direct counterpart in PROV-O, which models
provenance rather than authority.
CAE's Instrument kind is closer to deontic and legal ontology concepts
discussed below.

\subsection{Legal and Institutional Ontologies}

Modeling legal and institutional structures has been addressed by
several ontology efforts.
LKIF-Core~\cite{hoekstra2007lkif} provides an ontology for legal
knowledge, including norms, roles, and legal documents.
CAE's Instruments overlap conceptually with LKIF's normative concepts
but are more abstractly defined: an Instrument in CAE is any enduring
construct that grounds obligations, whether a statute, contract, permit,
or formally adopted program.

The Financial Industry Business Ontology (FIBO)~\cite{bennett2013fibo}
models contracts, parties, and obligations in the financial domain.
CAE shares FIBO's concern with obligations and parties but is
domain-neutral; where FIBO provides rich financial semantics,
CAE provides minimal cross-domain structure intended to support
interoperability across procurement, health, environment, education,
and other civic domains.

\subsection{Information Artifacts}

The Information Artifact Ontology (IAO)~\cite{ceusters2012iao}, developed
as an extension to BFO, models information-bearing entities such as
documents, data items, and specifications.
CAE's Instruments might be analyzed as information artifacts in the IAO
sense---they are often realized as documents (contracts, statutes,
permits) that carry normative content.
However, CAE treats Instruments as a primitive kind defined by their
functional role (grounding obligations and authority) rather than by
their material realization.
A single Instrument may be realized in multiple documents or amendments
over time; CAE abstracts over this multiplicity.

\subsection{Positioning CAE}

CAE is neither an upper ontology nor a domain ontology in the
traditional sense.
It occupies a middle position:
a small, stable set of entity kinds
designed to support interoperability across civic domains
without imposing the full apparatus of a foundational ontology or the
specificity of a sectoral model.

Readers familiar with foundational ontologies may recognize parallels
between CAE's enduring entities and time-indexed entities and distinctions
such as continuant/occurrent (BFO) or endurant/perdurant (DOLCE).
These references are provided for orientation;
CAE does not inherit, extend, or depend on any upper ontology.

The six CAE kinds are intended to be necessary and sufficient to
represent the structural relationships (obligations, authority,
action, measurement) that underlie civic accountability.
Richer semantics, domain vocabularies, and explanatory structures
are expected to be layered above CAE rather than incorporated into it.

While the foundational ontologies and standards reviewed above provide
essential grounding, none directly addresses
cross-domain civic accountability:
a minimal, stable partition of entity kinds sufficient to represent
obligations, authority, action, and measurement without imposing
domain-specific semantics or comprehensive legal modeling.
CAE is designed to fill this gap.
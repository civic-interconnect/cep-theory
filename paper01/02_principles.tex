% !TeX root = 00P1_cae_ontology.tex


\section{Design Principles and Scope}
\label{sec:principles}

The design of CAE is guided by a small number of explicit principles intended to
ensure clarity, stability, and extensibility.

\subsection{Accountability-Driven Inclusion}

CAE models entities only insofar as they participate in accountability-bearing
relationships.
An entity is introduced when it bears obligations, exercises
authority, participates in regulated actions, or is the subject of measurement
and evaluation.
CAE does not attempt to enumerate all organizations, facilities,
laws, or social phenomena exhaustively.

This exchange and obligation-driven inclusion rule prevents ontological bloat
and ensures that the ontology remains focused on structures that matter for civic
accountability~\cite{bowker2000sorting}.

In CEP, admissibility is determined by the presence of instrument-grounded events involving accountable actors.

\subsection{Disjoint Entity Kinds}

CAE defines a strict partition of entities into disjoint kinds.
Each entity is assigned exactly one kind, and entity kinds do not overlap.
This disjointness is
a foundational constraint: it prevents ambiguity and supports formal reasoning
over relationships.

Changes in function, responsibility, or context are represented through
relationships rather than reclassification.
An entity does not change kind over
time, even as its role within civic systems evolves.

\subsection{Roles as Relationships}

CAE does not model roles, sectors, or functions as entity kinds.
Concepts such as regulator, funder, operator, recipient, or subject of regulation are represented
as patterns of relationships among entities.
This approach avoids proliferation of role-specific entity types and
ensures that entity identity remains stable across contexts.

\subsection{Selective and Lazy Modeling}

Entities, particularly laws and regulations, are included selectively based on
their operational relevance.
Normative or regulatory instruments are introduced
only when they ground concrete events or observations.
This lazy modeling strategy supports scalable implementation
and avoids premature commitment to comprehensive legal or administrative catalogs.

\subsection{Neutrality and Separation of Concerns}

CAE is intentionally neutral with respect to causal inference, evaluative
judgment, and policy interpretation.
The ontology encodes structural
relationships that make accountability and outcomes inspectable,
but it does not assert that particular instruments cause particular
outcomes or that specific outcomes are desirable~\cite{pearl2009causality}.

Interpretation, explanation, and evidentiary reasoning are deferred to analytic
layers built upon CAE, including CEP and CEE.
This separation of concerns
supports transparency, reproducibility, and pluralistic analysis.

\subsection{Durability and Extensibility}

CAE is designed to remain stable over long time horizons.
The entity kinds and structural constraints are intended to be invariant even as institutions,
measurement practices, and data sources evolve~\cite{edwards2011infrastructure}.
Extension occurs through the addition of entities and relationships,
not through modification of the underlying ontology.

This design supports incremental adoption and cross-domain interoperability
without requiring coordinated changes across systems.
% !TeX root = 00P1_cae_ontology.tex

\clearpage
\section*{Appendix C. Glossary of Terms}
\addcontentsline{toc}{section}{Appendix C. Glossary of Terms}



This appendix provides concise definitions of key terms used throughout the paper.

\subsection*{Entity Kinds}

\textbf{Actor (A):}
An entity capable of bearing rights, obligations, or responsibilities within
civic systems when acting as an accountable party.
Examples include governments, public agencies, businesses, nonprofits, and universities.

\textbf{Site/Asset (S):}
A physical or operational entity that is acted upon but does not bear obligations.
Examples include facilities, buildings, infrastructure, and power plants.

\textbf{Instrument (I):}
An enduring construct that creates, modifies, or constrains rights, obligations,
or authority.
Examples include statutes, regulations, contracts, grants, permits, and licenses.

\textbf{Event (E):}
A time-indexed occurrence asserted under the authority of an Instrument.
Examples include payments, inspections, filings, violations, and audits.

\textbf{Jurisdiction (J):}
An entity that scopes authority, applicability, and governance, defining where
Instruments apply and where Events occur.
Examples include nations, states, municipalities, and regulatory regions.

\textbf{Observation (O):}
A measurement or indicator describing state, performance, or outcomes.
Observations do not create obligations or authorize actions.
Examples include health outcomes, coverage rates, emissions intensity measures,
and educational attainment indicators.

\subsection*{Design Concepts}

\textbf{Accountability Analysis:}
The examination of relationships, obligations, and authority structures within
civic systems in order to make responsibility and oversight inspectable.

\textbf{Accountability-bearing Relationship:}
A relationship that establishes or reflects accountability obligations between
entities, such as delegation of authority, participation in an Event, or the
measurement of outcomes.

\textbf{Applied Ontology:}
The use of ontological methods and principles to structure, clarify, and analyze
real-world domains for practical purposes.

\textbf{Competency Questions:}
Questions that an ontology should be able to answer, used to guide its development
and to evaluate its adequacy for the intended domain.

\textbf{Completeness:}
The extent to which an ontology captures the concepts and relationships required
for its stated purpose, without implying exhaustive coverage of all possible phenomena.

\textbf{Disjointness:}
The property that entity kinds do not overlap;
each entity belongs to exactly one kind.

\textbf{Domain Ontology:}
An ontology that captures concepts and relationships specific to a particular
application domain.
CAE is not a domain ontology.

\textbf{Domain-Constrained Reference Ontology:}
An ontology that provides a stable, reusable set of entity kinds and relationships
tailored to a particular domain, without committing to sector-specific taxonomies.
CAE is a domain-constrained reference ontology.

\textbf{Enduring Entity:}
An entity that persists through time while maintaining its identity, even as its
properties or relationships change.

\textbf{Entity Kind:}
A fundamental category of entities within the ontology, defined by distinct
identity criteria and invariant over time.

\textbf{Longitudinal Change:}
Variation or trends in data, conditions, or outcomes observed over extended periods
of time.

\textbf{Knowledge Representation:}
The formal specification of entities, relationships, and structures within a domain
to support interoperability and structured analysis.

\textbf{Ontology:}
A formal representation of entities within a domain
and the relationships that hold between them.

\textbf{Ontology Drift:}
The gradual divergence of an ontology's scope, structure, or commitments
from its original design intent.

\textbf{Selective Modeling:}
The inclusion of entities based on operational relevance rather than exhaustive
enumeration, introducing entities only when they participate in
accountability-bearing relationships.

\textbf{Semantics:}
The interpretation of structures and relationships defined by an ontology, without
implying evaluative or causal claims at the ontological level.

\textbf{Soundness:}
The property that an ontology's definitions and constraints are internally
consistent and aligned with their intended interpretations.

\textbf{Subclassing:}
The creation of a hierarchy of classes or categories within an ontology,
where more specific classes inherit properties and relationships from more
general ones.
CEA avoids subclassing within its six entity kinds.

\textbf{Time-Indexed Entity:}
An entity whose identity or assertions are associated with specific points or
intervals in time.

\textbf{Upper Ontology:}
A domain-independent ontology intended to provide general categories applicable
across many domains.
CAE is not an upper ontology.

\subsection*{Instrument Roles (Descriptive)}

\textbf{Normative role (descriptive):}
A functional role in which an Instrument establishes authority or defines
obligations.
Examples include statutes, acts, and treaties.

\textbf{Regulatory role (descriptive):}
A functional role in which an Instrument specifies procedures, thresholds, or
requirements.
Examples include regulations, rules, and administrative codes.


\subsection*{Methodological Context}

CAE is a formal ontology in the knowledge representation tradition:
a specification of what kinds of entities exist in the civic accountability domain,
what properties they have, and what relationships hold between them.
The six entity kinds form a strict partition: each entity belongs to exactly one
kind, and kinds do not overlap.
This structure supports rigorous, neutral reasoning about obligations, authority,
and evidence without embedding causal or evaluative assumptions.

% !TeX root = 00P1_cae_ontology.tex

\clearpage
\section*{Appendix A. Worked Examples}
\addcontentsline{toc}{section}{Appendix A. Worked Examples}

% ------------------------------------------------------------
\subsection*{A.1 Instrument Examples\footnotemark}
% ------------------------------------------------------------

\footnotetext{
    These examples are illustrative and do not define a controlled vocabulary
    or formal typology of Instruments.
}

\textbf{Scope note (non-normative):}
The labels used in this subsection (e.g., ``normative,'' ``programmatic,''
``procedural'') are descriptive only.
They do not introduce formal subclasses, types, or constraints on Instruments within CAE.
All entities listed here remain members of the single Instrument kind defined
in Section~\ref{subsec:instruments}.

These examples illustrate common kinds of authoritative documents and the general
instrumental function they represent.

A normative Instrument may be a high-level authoritative document that establishes
obligations, rights, or responsibilities.
Examples may include:
\begin{itemize}
    \item A federal statute enacted by Congress (e.g., the Clean Water Act).
    \item A state regulation promulgated by an environmental agency (e.g., air quality standards).
    \item A municipal ordinance adopted by a city council (e.g., zoning regulations).
    \item An international treaty ratified by national governments (e.g., Paris Agreement on climate change).
    \item A constitutional provision establishing fundamental rights or governance structures.
\end{itemize}

A programmatic Instrument may be an Instrument explicitly established by an authority
that creates standing obligations or eligibility conditions.
These exist independently of any single payment, award, or inspection, and
continue to ground accountability over time.
Examples may include:

\begin{itemize}
    \item A federal grant program established by statute or agency rule
          (e.g., a transportation infrastructure grant program with defined eligibility, reporting, and audit requirements).
    \item A state procurement program that defines allowable vendors, contracting thresholds, and compliance obligations.
    \item A public benefits program (housing assistance, health coverage, educational aid)
          with formally defined enrollment rules and verification requirements.
    \item A research funding program operated by a public agency with binding reporting, data-sharing, or compliance conditions.
    \item A regulatory compliance program (e.g., emissions reporting, safety certification)
          that structures recurring obligations across many Events.
\end{itemize}

A formally adopted procedural regime may be a procedural Instrument that is binding because it is adopted.
These constrain how decisions or actions must be carried out, not just what is allowed.

Examples may include:
\begin{itemize}
    \item A procurement evaluation procedure formally adopted by an agency (e.g., a scoring rubric mandated by policy or regulation).
    \item A compliance review process required by regulation (e.g., mandatory inspection sequences or escalation rules).
    \item A grant review workflow codified in agency policy that determines eligibility, ranking, and award decisions.
    \item A data reporting protocol mandated by regulation (e.g., required schemas, submission cycles, validation steps).
    \item A formally adopted analytical methodology when explicitly named in statute, rule, or contract
          (e.g., a required risk-scoring method used for eligibility or enforcement).
\end{itemize}

Many Instruments combine multiple functional aspects.
For example, a statute may authorize a program, mandate procedures,
and define reporting obligations.
CAE does not require assigning an Instrument to a single functional pattern;
the examples above highlight dominant characteristics only.
Formal typologies of Instruments are outside the scope of this paper.

This appendix also provides concrete examples illustrating how CAE's six entity
kinds and relationship constraints apply across different civic domains.
The goal is to demonstrate that the ontology is sufficient for modeling
diverse accountability scenarios without domain-specific extensions.

The following examples use real-world entities purely for illustrative purposes.
No evaluative, legal, or causal claims are implied beyond the structural relationships shown.

% ------------------------------------------------------------
\subsection*{A.2 Chicago Procurement: Following Federal Funds to Local Contractors}
% ------------------------------------------------------------


\textbf{Scenario:}
The Federal Highway Administration (FHWA) provides funding to the Illinois
Department of Transportation (IDOT), which contracts with a construction
company to repair Chicago infrastructure.

\textbf{CAE Representation:}

\begin{mdframed}
    \textbf{Actors (A):}
    \begin{itemize}[nosep]
        \item FHWA (federal agency)
        \item IDOT (state agency)
        \item ABC Construction Company (contractor)
        \item City of Chicago (local government)
    \end{itemize}

    \textbf{Instruments (I):}
    \begin{itemize}[nosep]
        \item Federal Highway Act (normative Instrument)
        \item IDOT-FHWA Grant Agreement (programmatic Instrument)
        \item Construction Contract (programmatic Instrument)
    \end{itemize}

    \textbf{Events (E):}
    \begin{itemize}[nosep]
        \item Payment from FHWA to IDOT (\$2.5M, 2024-03-15)
        \item Payment from IDOT to ABC Construction (\$2.4M, 2024-04-01)
        \item Project completion inspection (2024-11-20)
    \end{itemize}

    \textbf{Sites (S):}
    \begin{itemize}[nosep]
        \item Bridge on Interstate 90 (infrastructure site)
    \end{itemize}

    \textbf{Jurisdictions (J):}
    \begin{itemize}[nosep]
        \item United States (federal scope)
        \item Illinois (state scope)
        \item Chicago (municipal scope)
    \end{itemize}

    \textbf{Observations (O):}
    \begin{itemize}[nosep]
        \item Bridge safety rating (2025-01-10: "Good")
        \item Project completion rate (100\%
        )
    \end{itemize}
\end{mdframed}

\textbf{Key Relationships:}
\begin{itemize}[nosep]
    \item \emph{enacts}: Congress $\rightarrow$ Federal Highway Act
    \item \emph{implements}: IDOT-FHWA Grant $\rightarrow$ Federal Highway Act
    \item \emph{party-to}: FHWA $\rightarrow$ IDOT-FHWA Grant
    \item \emph{party-to}: IDOT $\rightarrow$ IDOT-FHWA Grant
    \item \emph{occurs-under}: Payment Event $\rightarrow$ IDOT-FHWA Grant
    \item \emph{acts-on}: Inspection Event $\rightarrow$ I-90 Bridge
    \item \emph{located-in}: I-90 Bridge $\rightarrow$ Chicago
    \item \emph{measures}: Safety Rating $\rightarrow$ I-90 Bridge
\end{itemize}

\textbf{Note:} This example demonstrates selective inclusion.
Many other federal programs, state agencies, and contractors exist but are not modeled
unless they participate in accountability-bearing relationships relevant to
this exchange.

% ------------------------------------------------------------
\subsection*{A.3 Public Health Program: Statute to Population Outcomes}
% ------------------------------------------------------------


\textbf{Scenario:}
The Affordable Care Act (ACA) authorizes the Department of Health and Human
Services (HHS) to expand Medicaid.
States adopt the expansion, 
health insurance coverage changes over time, 
with longitudinal changes observed in population health indicators.

\textbf{CAE Representation:}

\begin{mdframed}
    \textbf{Actors (A):}
    \begin{itemize}[nosep]
        \item U.S. Congress
        \item HHS (federal agency)
        \item California Department of Health Care Services (state agency)
        \item Kaiser Permanente (health insurer)
    \end{itemize}

    \textbf{Instruments (I):}
    \begin{itemize}[nosep]
        \item Affordable Care Act (normative Instrument)
        \item Medicaid Expansion Regulation (regulatory Instrument)
        \item California Medicaid Plan (programmatic Instrument)
    \end{itemize}

    \textbf{Events (E):}
    \begin{itemize}[nosep]
        \item California adopts Medicaid expansion (2014-01-01)
        \item Enrollment event (2014-03-15, 500,000 new enrollees)
        \item Claims payments (ongoing, monthly)
    \end{itemize}

    \textbf{Jurisdictions (J):}
    \begin{itemize}[nosep]
        \item United States
        \item California
    \end{itemize}

    \textbf{Observations (O):}
    \begin{itemize}[nosep]
        \item Uninsured rate (2013: 17\%, 2020: 7\%)
        \item Preventable hospitalizations (2013: 45 per 1000, 2020: 32 per 1000)
        \item Life expectancy (measured at county level, longitudinal)
    \end{itemize}
\end{mdframed}

\textbf{Temporal Structure:}
Events occur at specific times (2014-01-01, 2014-03-15), while Observations
track outcomes over years (2013-2020).
The ACA (Instrument) remains an enduring entity throughout this period.

\textbf{Accountability Chain:}
Observations are connected to the ACA through shared Jurisdictions and temporal
overlap, enabling analysis of health outcomes before and after expansion.
CAE does not assert that the ACA caused the improvements, but it makes the
structural connections explicit for hypothesis testing.

% ------------------------------------------------------------
\subsection*{A.4 Environmental Regulation: Permits and Emissions}
% ------------------------------------------------------------


\textbf{Scenario:}
The Clean Air Act authorizes the EPA to regulate emissions.
A state environmental agency issues a permit to a power plant.
The plant submits emissions reports, and air quality is monitored.

\textbf{CAE Representation:}

\begin{mdframed}
    \textbf{Actors (A):}
    \begin{itemize}[nosep]
        \item EPA (federal regulator)
        \item Ohio EPA (state regulator)
        \item FirstEnergy (power plant operator)
    \end{itemize}

    \textbf{Instruments (I):}
    \begin{itemize}[nosep]
        \item Clean Air Act (normative Instrument)
        \item Title V Operating Permit (regulatory Instrument)
    \end{itemize}

    \textbf{Events (E):}
    \begin{itemize}[nosep]
        \item Permit issuance (2020-06-01)
        \item Quarterly emissions report (2024-10-01)
        \item Compliance inspection (2024-11-15)
    \end{itemize}

    \textbf{Sites (S):}
    \begin{itemize}[nosep]
        \item W.H. Sammis Power Plant (coal facility)
    \end{itemize}

    \textbf{Jurisdictions (J):}
    \begin{itemize}[nosep]
        \item United States
        \item Ohio
        \item Stratton Air Basin (regulatory region)
    \end{itemize}

    \textbf{Observations (O):}
    \begin{itemize}[nosep]
        \item \(O_{\mathrm{SO_2}}^{\mathrm{intensity}}(2024\text{Q3}) = 0.12 \;\text{lb/MMBtu}\)
        \item \(O_{\mathrm{PM}_{2.5}}^{\mathrm{ambient}} = 8.5 \;\mu\text{g/m}^3\)
    \end{itemize}
\end{mdframed}

\textbf{Jurisdictional Nesting:}
The Stratton Air Basin (environmental jurisdiction) overlaps with Ohio
(political jurisdiction).
Both are represented as distinct Jurisdictions with explicit \emph{contains} relationships.

\textbf{Actor-Site Distinction:}
FirstEnergy (Actor) operates the Sammis Plant (Site).
The permit constrains the Site's operations, while FirstEnergy bears the legal obligation to comply.

% ------------------------------------------------------------
\subsection*{A.5 Infrastructure Investment: Long-Term Public Value}
% ------------------------------------------------------------


\textbf{Scenario:}
A state issues bonds to fund rural broadband infrastructure.
Private ISPs are awarded contracts to build fiber networks.
Over time, internet access and economic indicators improve.

\textbf{CAE Representation:}

\begin{mdframed}
    \textbf{Actors (A):}
    \begin{itemize}[nosep]
        \item Minnesota Department of Employment and Economic Development (state agency)
        \item Arvig Communications (ISP contractor)
        \item Local counties (grant recipients)
    \end{itemize}

    \textbf{Instruments (I):}
    \begin{itemize}[nosep]
        \item Minnesota Broadband Development Act (normative Instrument)
        \item Bond authorization (programmatic Instrument)
        \item Construction grant (programmatic Instrument)
    \end{itemize}

    \textbf{Events (E):}
    \begin{itemize}[nosep]
        \item Bond issuance (2018-03-01, \$100M)
        \item Grant award (2018-06-01, \$5M to Arvig)
        \item Network completion (2020-12-01)
    \end{itemize}

    \textbf{Sites (S):}
    \begin{itemize}[nosep]
        \item Fiber optic network (rural Minnesota)
    \end{itemize}

    \textbf{Jurisdictions (J):}
    \begin{itemize}[nosep]
        \item Minnesota
        \item Clay County, Norman County (rural counties)
    \end{itemize}

    \textbf{Observations (O):}
    \begin{itemize}[nosep]
        \item Broadband access rate (2018: 45\%, 2023: 92\%)
        \item Median household income (2018: \$52K, 2023: \$58K)
        \item Remote work adoption (2019: 8\%, 2023: 24\%)
    \end{itemize}
\end{mdframed}

\textbf{Longitudinal Analysis:}
Observations span 5+ years, tracking outcomes well beyond the initial
investment Events.
This structure supports analysis of public value that
accrues over decades, not just immediate financial returns.

\textbf{Diffuse Benefits:}
Unlike direct procurement (Example B.1), infrastructure benefits are diffuse
and population-level.
Observations capture these outcomes explicitly rather
than inferring them from transaction Events.

% ------------------------------------------------------------
\subsection*{A.6 Role Multiplicity: University as Multi-Role Actor}
% ------------------------------------------------------------


\textbf{Scenario:}
A university receives federal research grants, operates regulated laboratories,
employs faculty researchers, and reports compliance data.

\textbf{CAE Representation:}

\begin{mdframed}
    \textbf{Single Actor (A):}
    \begin{itemize}[nosep]
        \item University of Minnesota
    \end{itemize}

    \textbf{Multiple Relationship Patterns:}
    \begin{itemize}[nosep]
        \item \emph{party-to}: University $\rightarrow$ NSF Grant (role: grant recipient)
        \item \emph{operates}: University $\rightarrow$ Biosafety Lab (role: facility operator)
        \item \emph{employs}: University $\rightarrow$ Researcher (role: employer)
        \item \emph{subject-of}: Compliance Inspection $\rightarrow$ University (role: regulated entity)
    \end{itemize}
\end{mdframed}

\textbf{Role as Relationship Pattern:}
The university is a single Actor.
Its roles (recipient, operator, employer, regulated entity)
emerge from its relational position with respect to
Instruments, Sites, and Events.
No reclassification is needed when roles change.

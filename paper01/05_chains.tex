% !TeX root = 00P1_cae_ontology.tex

\section{Laws, Regulations, and Accountability Chains}
\label{sec:laws}

This section describes how laws and regulations are represented within the CAE
ontology and how they participate in accountability chains without requiring
exhaustive modeling of legal texts.
CAE treats statutes and regulations as Instruments that ground 
authority, obligations, and enforcement while remaining
structurally compatible with exchange and evidence semantics.

The objective is not to encode legal doctrine, but to preserve the causal and
accountability structure through which public authority produces concrete
actions and observable outcomes.

\subsection{Normative and Regulatory Instruments}
\label{subsec:normative}

Within CAE, laws and regulations are modeled uniformly as Instruments.
A statute, act, or treaty is treated as a normative instrument: an enduring
construct that establishes authority, delegates power, or defines obligations at
a high level.
Regulations, rules, and administrative codes are treated as
regulatory instruments: instruments that operationalize normative instruments by
specifying procedures, thresholds, or reporting requirements.

Both normative and regulatory instruments share the defining property of
grounding accountability relationships.
They differ only in scope and level of abstraction, not in ontological kind.
This approach avoids proliferating entity types while preserving 
the hierarchical structure of legal authority.

\subsection{Delegation and Implementation Chains}
\label{subsec:delegation}

Accountability chains emerge through explicit relationships between instruments
and actors.
A normative instrument may delegate authority to one or more Actors,
which in turn issue regulatory or programmatic instruments.
These downstream instruments constrain Events and give rise to Observations.

For example, a statute enacted by a legislature delegates authority to an agency.
The agency issues regulations implementing the statute, which authorize permits
governing the operation of facilities.
Compliance reports, inspections, and violations occur as Events under these permits, 
and population-level impacts are captured as Observations.

Representing delegation and implementation explicitly allows CAE to trace how
authority flows from abstract law to concrete outcomes without conflating
normative intent with execution.

\subsection{Selective Inclusion and Lazy Modeling}
\label{subsec:selective}

CAE does not require comprehensive inclusion of all laws or regulations.
Normative and regulatory instruments are introduced only when they ground
accountability-bearing Events or Observations.
This selective inclusion strategy prevents ontological bloat and supports scalable implementation.

For instance, a health statute need not be modeled until insurance programs,
payments, enforcement actions, or outcome measurements associated with that
statute are introduced.
The presence of an instrument in CAE is therefore driven
by its operational relevance rather than by its legal prominence.

This design ensures that CAE remains extensible and performant while preserving
the structural relationships necessary for downstream semantics.

\subsection{Jurisdictional Scope and Applicability}
\label{subsec:scope}

Every normative or regulatory instrument applies within one or more
Jurisdictions.
Jurisdictional scope is represented explicitly through
relationships rather than being implied by instrument identity.
This allows instruments with overlapping or nested applicability to coexist without ambiguity.

Explicit jurisdictional modeling supports comparative analysis across regions and
time, enabling the study of how differing legal regimes relate to variations in
events and observations.
It also permits instruments to evolve or be superseded
without altering the identity of affected entities.

\subsection{Accountability Chains and Observability}
\label{subsec:chains}

An accountability chain is defined as a connected subgraph linking normative or
regulatory instruments to concrete events and observations through delegation,
issuance, participation, and measurement relationships.
Such chains make explicit how public authority is exercised and how its consequences are observed.

CAE does not assert causal claims within these chains.
Instead, it provides a structural representation that allows causal hypotheses to be articulated,
tested, or debated using evidence attached through CEE.
This separation of structure from inference ensures neutrality while enabling rigorous analysis.

\subsection{Implications for Longitudinal Analysis}
\label{subsec:longitudinal}

By modeling laws and regulations as enduring instruments within accountability
chains, CAE supports longitudinal analysis of public interventions.
Changes in legal regimes, regulatory thresholds, or enforcement practices can be represented
as modifications or additions to instruments and relationships rather than as
redefinitions of entities.

This approach enables comparative study of long-term public investments and
policy choices, such as infrastructure development, public health interventions,
or environmental regulation, while preserving continuity of entity identity
across time.
% !TeX root = 00P1_cae_ontology.tex

\clearpage
\section*{Appendix E. Glossary of CAE Terms}
\addcontentsline{toc}{section}{Appendix E. Glossary of CAE Terms}

This appendix provides definitions of key terms used throughout the paper
to make CAE accessible to readers from public administration, civic technology,
data engineering, and policy analysis communities.

\textbf{Accountability Chain:}
A connected subgraph linking normative instruments to concrete events and
observations through delegation, issuance, participation, and measurement
relationships.
Makes explicit how public authority is exercised and consequences observed.

\textbf{Actor (A):}
An entity capable of bearing rights, obligations, or responsibilities within
civic systems.
Examples: governments, agencies, businesses, nonprofits, universities.

\textbf{Asset/Site (S):}
A physical or operational entity that is acted upon but does not bear obligations.
Examples: facilities, buildings, infrastructure, power plants.

\textbf{Event (E):}
A time-indexed occurrence asserted under the authority of an Instrument.
Examples: payments, inspections, filings, violations, audits.

\textbf{Instrument (I):}
An enduring construct that creates, modifies, or constrains rights, obligations,
or authority.
Examples: statutes, regulations, contracts, grants, permits, licenses.

\textbf{Jurisdiction (J):}
An entity that scopes authority, applicability, and governance.
Defines where
Instruments apply and Events occur.
Examples: nations, states, municipalities, regulatory regions.

\textbf{Observation (O):}
A measurement or indicator describing state, performance, or outcomes.
Does not create obligations.
Examples: health outcomes, coverage rates, emissions intensity, educational attainment.

\textbf{Lazy Modeling:}
The selective inclusion of entities based on operational relevance rather than
exhaustive enumeration.
Instruments introduced only when they ground concrete events or observations.

\textbf{Normative Instrument:}
A high-level Instrument that establishes authority or defines obligations.
Examples: statutes, acts, treaties.

\textbf{Regulatory Instrument:}
An Instrument that operationalizes normative instruments by specifying procedures,
thresholds, or requirements.
Examples: regulations, rules, administrative codes.
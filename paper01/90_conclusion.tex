% !TeX root = 00P1_cae_ontology.tex

\section{Discussion and Limitations}
\label{sec:discussion}

CAE is intentionally conservative in its ontological commitments.
It does not attempt to enumerate all possible civic entities, nor does it claim completeness
with respect to real-world institutional complexity.
Entities are introduced only insofar as they participate in accountability-bearing relationships, and
many socially significant phenomena remain outside the scope of the ontology.

The framework does not encode causal claims, evaluative judgments, or policy prescriptions.
While CAE enables structural comparison and longitudinal analysis,
interpretation of outcomes and attribution of causality are deferred to analytic
layers built upon the ontology.
This design choice prioritizes neutrality and reusability over prescriptive modeling.

CAE also depends on the quality and availability of upstream data sources.
Incomplete, inconsistent, or biased data may limit the conclusions that can be
drawn from analyses grounded in the ontology.
However, by making structural relationships explicit,
CAE enables such limitations to be surfaced rather than hidden.

Finally, while CAE is designed to be stable over time, real-world institutions,
laws, and measurement practices evolve.
The ontology accommodates such evolution through the addition or modification of entities and relationships without
requiring reclassification or type mutation.
This approach supports incremental extension while preserving semantic continuity.

The adequacy of CAE is evaluated using competency questions
grounded in real civic datasets rather than axiomatic completeness.

\section{Conclusion}
\label{sec:conclusion}

This paper has introduced CAE, a formal ontology of civic accountable
entities designed to represent the structural relationships underlying
civic accountability.
By defining a strict, disjoint partition of entity kinds and
constraining admissible relationships, CAE establishes a stable foundation
for modeling obligations, authority, actions, and outcomes.

The ontology emphasizes accountability as the organizing principle,
enabling representation of long-term public investments, regulatory regimes,
financial flows, and measured outcomes without collapsing structure into
interpretation.
By separating ontological commitment from domain-specific vocabularies and
analytic methods, CAE supports interoperability, auditability, and
longitudinal analysis across jurisdictions and time.

CAE is designed to be extensible, neutral, and durable.
It supports downstream frameworks for modeling exchanges and attaching
structured explanations to civic decisions, without requiring modification
of the ontological kinds defined here.
The six entity kinds and their structural constraints provide a basis for
making civic systems and their outcomes inspectable, comparable, and accountable.

\section*{Acknowledgements}

Portions of this work were developed through human-computer collaboration
using modern computational tools.
Generative language models were used to assist with editing, formatting,
and consistency checking during manuscript preparation.
All conceptual framing, formal development, results, interpretations, and conclusions
are the author's own.
All generated suggestions were critically reviewed and validated, and
the author takes full responsibility for the content of this work.

% !TeX root = 00P1_cae_ontology.tex

\section{Ontological Partition of Accountable Entities}
\label{sec:ontology}

This section defines the core ontological commitment of the Civic Accountable
Entities (CAE) framework.
CAE introduces a strict partition of accountable entities into six disjoint kinds:
Actors (A), Sites/Assets (S), Instruments (I),
Events (E), Jurisdictions (J), and Observations (O).
Each entity instantiated within CAE is assigned exactly one kind.
Entity kinds are invariant over time:
entities do not change kind, and apparent role changes are represented through
relationships rather than reclassification.

The partition is designed to support formal reasoning about obligations, authority,
and evidence while preventing ontological overlap.
Inclusion of an entity in CAE is driven by participation in
obligations, authority relationships, or accountability-bearing exchanges,
rather than by descriptive completeness or sectoral classification.

\subsection{Actors (A)}
\label{subsec:actors}

Actors are entities capable of bearing rights, obligations, or responsibilities
within civic systems.
An Actor may initiate, receive, authorize, or be held
accountable for actions governed by Instruments and manifested through Events.
Actors are the only entity kind that may serve as parties to obligations.

Examples include governments, public agencies, private businesses,
nonprofit organizations, universities, research institutes, and other organizational bodies
that participate in civic accountability relationships.
Public or private status, sector, mission, and organizational role are treated
as attributes or relationships, not as entity kinds.

An Actor is introduced into CAE only when it participates in an accountability
relationship, such as receiving funds, issuing authority, operating regulated
Sites, or being subject to reporting or enforcement.
CAE does not attempt to enumerate organizations exhaustively;
it models Actors only insofar as they are implicated in
obligations, authority, or accountability-bearing exchanges.

See Section~\ref{subsec:jurisdictions} for clarification on Actors and Jurisdictions with shared labels.

\subsection{Sites/Assets (S)}
\label{subsec:sites}

Sites/Assets are physical or operational entities that are acted upon but do
not themselves bear obligations.
They provide the spatial, infrastructural, or
material substrate upon which civic activity occurs.
Sites/Assets may be owned, operated, regulated, inspected, or measured,
but they are not parties to Instruments.

Examples include facilities, buildings, campuses, power plants, laboratories,
stores, transportation infrastructure, and other physical or operational installations.
Geographic location is an intrinsic property of Sites/Assets and provides a
natural point of attachment to Jurisdictions.

Treating Sites/Assets as a distinct entity kind ensures a clear separation
between accountable actors and the physical or operational entities through
which obligations are exercised or impacts are realized.

\subsection{Instruments (I)}
\label{subsec:instruments}

Instruments are \emph{enduring} constructs:
they persist over time and are not tied to a single occurrence or timestamp,
even though they may be enacted, amended, applied, or terminated by Events.

Instruments create, modify, delegate, or constrain
rights, obligations, or authority.
Instruments mediate relationships between Actors
and govern the conditions under which Events may occur.

An Instrument provides the normative or regulatory basis
for why an action, decision, or outcome
takes the form it does,
independent of the specific Event that realizes it.
By distinguishing Instruments from Events,
CAE separates the existence of obligations or authority
from their execution, fulfillment, or violation.

CAE distinguishes common functional roles of Instruments
without introducing additional entity kinds.

Examples include formal agreements (such as contracts and memoranda of understanding),
statutes and laws,
rules and regulations,
and programmatic constructs such as grants, permits, licenses,
formally constituted programs, or formally adopted procedural regimes.
These examples are illustrative rather than exhaustive;
all Instruments share the defining property of grounding
obligations or authority that can give rise to accountable Events.

Formally constituted programs and procedural regimes include only
those that are explicitly adopted by an authoritative body
and impose binding conditions on participation, evaluation, or compliance.

In CAE, a grant program or award framework is modeled as an Instrument,
while the disbursement of grant funds is modeled as an Event.

CAE deliberately excludes artifacts that do not themselves create or constrain
obligations.
Operational tools, analytical software, formatting utilities,
quality assurance tools, or internal workflows are not treated as Instruments
unless they are explicitly incorporated into
a binding normative, regulatory, or contractual structure.
This constraint preserves the stability of the ontology
and prevents implementation-specific mechanisms from being mistaken
for sources of civic authority.

Instruments are included in CAE only when they
give rise to concrete Events or Observations.
This design avoids exhaustive modeling of legal or administrative texts
while preserving the accountability structure required for tracing
downstream exchanges and evidentiary relationships.

\subsection{Events (E)}
\label{subsec:events}

Events are time-indexed occurrences that are recorded or asserted
within the scope of one or more Instruments.
An Event records that something happened at a particular time and place and may
involve one or more Actors, Sites, or Jurisdictions.
Events constitute the primary evidence of activity within civic systems.

Examples include payments, inspections, filings, emissions submissions,
audits, enforcement actions, and other discrete occurrences,
including the execution, violation, or fulfillment of obligations.

Events are not enduring objects;
their identity is inseparable from their temporal occurrence and provenance.

By separating Events from Instruments, CAE distinguishes between the existence
of obligations and the activities that occur under, in response to,
or in violation of those obligations.
This separation is essential for representing compliance, non-compliance,
performance, and accountability over time.

\subsection{Jurisdictions (J)}
\label{subsec:jurisdictions}

Jurisdictions are entities that scope authority, applicability, and governance.
They define where Instruments apply, where Events may occur, and how Observations are interpreted.
Jurisdictions are not Actors; they do not initiate actions or bear obligations.

Examples include nations, states, provinces, municipalities, regulatory regions,
air basins, and watersheds.
Jurisdictions may be nested or overlapping, and such
structure is represented explicitly through relationships.

Treating Jurisdictions as a distinct entity kind allows CAE to model legal,
regulatory, and environmental scope without conflating authority with agency or action.

\paragraph{Actors and Jurisdictions as Distinct Entities.}

In CAE, political or administrative names (e.g., California, City of Chicago)
may refer to multiple distinct entities that occupy different ontological roles.
When such an entity acts, e.g. by entering contracts, issuing permits, making payments,
or bearing obligations, it is modeled as an \emph{Actor}.
When the same named entity defines legal scope, authority, or applicability,
it is modeled as a \emph{Jurisdiction}.
These are distinct entities with separate identities, even when they share
a common label or geographic extent.

This separation ensures disjointness between Actors and Jurisdictions and
prevents role-based reclassification.
An Actor may operate within, be constrained by, or exercise authority over
a Jurisdiction, but it does not become a Jurisdiction by acting,
nor does a Jurisdiction become an Actor by scoping authority.


\subsection{Observations (O)}
\label{subsec:observations}

Observations are measurements or indicators describing the state, performance,
or outcomes associated with Actors, Sites, Instruments, Events, or Jurisdictions.
Observations do not create obligations and do not represent actions; they assert
measured or derived facts with associated provenance.

Examples include health outcomes, emissions intensity metrics,
educational attainment measures, coverage metrics, and other longitudinal or
comparative indicators.
Observations may be aggregated, statistical, or model-based
and are typically associated with populations or regions through attributes and
relationships.

Introducing Observations as a first-class entity kind enables CAE to represent
long-term public value and outcomes without collapsing measurement into
Events or Instruments.
This separation supports comparison across time
and jurisdictions while remaining neutral with respect to causal interpretation.


\subsection{Ontological Stability and Non-Goals}

CAE is intentionally minimal and non-exhaustive.
It does not aim to provide sector taxonomies,
domain-specific subclasses, or comprehensive classifications of civic activity.
The six entity kinds defined are intended to be necessary and sufficient to
ground accountability, exchange, and evidence across domains,
independent of sector, policy area, or implementation technology.
Future extensions are expected to occur through
vocabularies, schemas, and domain-specific layers built atop CAE,
rather than through modification or proliferation of the ontological kinds themselves.
Stability of the CAE partition over time is a design goal:
changes in practice, tooling, or policy should be representable
through new entities and relationships,
not through alteration of the underlying ontology.

By enforcing disjointness among entity kinds, CAE prevents category confusion
and ensures clarity in modeling accountability relationships.
This clarity is essential for formal reasoning about obligations,
authority, and evidence within civic systems.

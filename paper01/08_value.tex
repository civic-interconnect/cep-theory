% !TeX root = 00P1_cae_ontology.tex

\section{Outcomes, Observations, and Public Value}
\label{sec:outcomes}

This section addresses how CAE represents outcomes and public value through
Observations, enabling longitudinal and comparative analysis of civic systems
without embedding normative judgments or causal assertions in the ontology
itself.
CAE treats outcomes as observable properties of actors, sites, or
jurisdictions over time, grounded in accountability chains but analytically
distinct from actions and obligations.

By separating structure from interpretation, CAE enables outcomes to be examined
as inspectable evidence rather than as implicit conclusions.

\subsection{Observations as First-Class Entities}
\label{subsec:observations-first}

Observations are first-class entities within CAE, distinct from Events and
Instruments.
An Observation represents a measurement, indicator, or statistical
assertion describing the state, performance, or outcome associated with one or
more accountable entities.
Observations do not create obligations, authorize
actions, or record occurrences; they assert measured or derived facts with
explicit provenance.

Examples include public health metrics, infrastructure coverage rates, emissions
intensity measures, food access indicators, educational attainment statistics,
and economic or environmental indices.
Observations may be derived from surveys, administrative data, sensors, or analytic models, and may be reported at varying
levels of aggregation.

Treating Observations as a distinct entity kind prevents the conflation of
measurement with action and ensures that outcome data can evolve independently
of the instruments or events that give rise to it.

\subsection{Attachment and Scope of Observations}
\label{subsec:attachment}

Observations may be associated with Actors, Sites, Jurisdictions, or defined
populations through typed relationships.
For example, a health outcome observation may be associated with a jurisdiction
and a demographic group, while an emissions intensity observation may be
associated with a site or facility.

Temporal scope is an intrinsic property of Observations, enabling representation
of trends, baselines, and changes over time.
Spatial and jurisdictional scope are represented explicitly, allowing comparable observations to coexist across
regions with differing legal or institutional contexts.

This explicit attachment enables comparative analysis without requiring
redefinition of entity identity or reclassification of entity kinds.

\subsection{From Accountability Chains to Outcome Analysis}
\label{subsec:chains-outcomes}

Observations are connected to accountability chains through relationships to
Events, Instruments, and Jurisdictions.
These connections make explicit which instruments and actions are relevant to a given outcome without asserting that
any particular instrument caused the observed result.

For example, public health outcomes may be associated with vaccination programs,
regulatory regimes, and funding events through shared jurisdictions and temporal
overlap.
Infrastructure access observations may be linked to investment
programs, construction events, and regulatory requirements governing service
provision.

By representing these connections structurally, CAE enables analysts to examine
patterns, correlations, and hypotheses while preserving neutrality with respect
to causation.

\subsection{Public Value and Long-Term Investments}
\label{subsec:public-value}

Many of the most consequential civic investments involve high upfront costs and
long-term, diffuse benefits.
Examples include public health interventions,
communications infrastructure, transportation systems, education, and basic
scientific research.
Such investments often resist evaluation under short-term accounting frameworks.

CAE supports representation of public value by allowing Observations to capture
longitudinal outcomes such as population health, access to services, economic
mobility, environmental quality, and quality of life.
These outcomes can be examined alongside the instruments and events that structure investment and
governance, enabling analysis that extends beyond immediate financial return.

This capability is essential for examining effectiveness and efficiency of
public spending without reducing value to short-term metrics alone.

\subsection{Neutrality and Interpretive Separation}
\label{subsec:neutrality}

CAE deliberately avoids encoding evaluative judgments or causal conclusions in
the ontology.
Whether an outcome is considered desirable, effective, or
efficient is a matter of interpretation, policy choice, or analysis external to
the ontological layer.

By providing a structured representation of entities, relationships, and
observations, CAE enables such interpretations to be made explicit, contested,
and revised.
This separation of representation from judgment supports
transparency, reproducibility, and pluralistic analysis.

\subsection{Implications for Comparative and Global Analysis}
\label{subsec:comparative}

The explicit representation of Observations, Jurisdictions, and accountability
chains enables comparative analysis across regions, institutions, and time.
Differences in legal regimes, investment strategies, or governance structures can
be examined alongside corresponding variations in outcomes without requiring
domain-specific ontological extensions.

This design supports global comparison of public interventions, infrastructure
development, and social outcomes, making long-term effects inspectable even in
the presence of incomplete or heterogeneous data sources.
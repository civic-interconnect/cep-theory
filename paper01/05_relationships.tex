% !TeX root = 00P1_cae_ontology.tex

\section{Relationships and Structural Constraints}
\label{sec:relationships}

Having defined the disjoint kinds of civic accountable entities, we now specify
the allowed relationships between them and the structural constraints that
govern their valid configuration.
CAE does not enumerate all possible relationships;
instead, it constrains the space of valid relationships so that accountability,
authority, and evidence can be expressed without ontological ambiguity.

Relationships in CAE are typed, directional, and kind-constrained.
They serve as the primary carriers of semantic meaning, while entity kinds remain invariant
over time.
Roles, classifications, and contextual interpretations are expressed
through relationships and attributes rather than through reclassification of
entities.

\subsection{Kind-Constrained Relationships}
\label{subsec:kind-constrained}

Each relationship in CAE specifies admissible source and target kinds.
These constraints prevent category errors such as treating physical Sites as
obligation-bearing parties or conflating Events with enduring authority.

Representative examples include:
\begin{itemize}[nosep]
    \item \emph{enacts} : Actor $\rightarrow$ Instrument
    \item \emph{implements} : Instrument $\rightarrow$ Instrument
    \item \emph{issues} : Actor $\rightarrow$ Instrument
    \item \emph{party-to} : Actor $\rightarrow$ Instrument
    \item \emph{occurs-under} : Event $\rightarrow$ Instrument
    \item \emph{involves} : Event $\rightarrow$ Actor
    \item \emph{acts-on} : Event $\rightarrow$ Site
    \item \emph{located-in} : Site $\rightarrow$ Jurisdiction
    \item \emph{applies-in} : Instrument $\rightarrow$ Jurisdiction
    \item \emph{measures} : Observation $\rightarrow$ (Actor $\mid$ Site $\mid$ Jurisdiction)
\end{itemize}

These constraints ensure that accountability flows through relationships rather
than being implicit in entity types.

\subsection{Authority and Obligation Flow}
\label{subsec:authority}

CAE encodes authority and obligation as relational structure rather than as
intrinsic properties of entities.
Authority originates with normative or regulatory Instruments
enacted or issued by Actors and flows through relationships
to constrain Events and generate Observations.

For example, a statute (Instrument) enacted by a legislature (Actor) delegates
authority to an agency (Actor), which issues permits (Instruments) governing the
operation of facilities (Sites).
Emissions reports (Events) occur under these permits, and health outcomes (Observations) are measured within relevant
Jurisdictions.

This directional flow preserves a clear distinction between the existence of
authority, its execution, and its observed consequences.

\subsection{Temporal Structure and Provenance}
\label{subsec:temporal}

Time is represented explicitly only through Events and Observations.
Actors, Sites, Instruments, and Jurisdictions are enduring entities whose identity is not
defined by temporal occurrence, although they may participate in time-indexed
relationships.

Every Event and Observation is associated with provenance information specifying
its source, time, and evidentiary context.
CAE itself does not prescribe a provenance model;
rather, it ensures that provenance can be attached without
conflicting with entity kinds or relationship constraints.

This separation supports longitudinal analysis while avoiding the reification of
temporal states as distinct entities.

\subsection{Role Representation}
\label{subsec:roles}

Roles are not modeled as entity kinds within CAE.
Instead, roles emerge from patterns of relationships.
An Actor may simultaneously occupy multiple roles,
such as regulator, funder, operator, or recipient, depending on its relational
position with respect to Instruments and Events.

For example, a university may act as a grant recipient, a facility operator, and
a reporting entity without requiring reclassification.
This approach avoids role-based overlap and ensures that
entity identity remains stable even as context changes.

\subsection{Structural Invariants}
\label{subsec:invariants}

CAE enforces several structural invariants:
\begin{itemize}[nosep]
    \item Each entity belongs to exactly one entity kind.
    \item Entity kinds do not change over time.
    \item Authority and obligation are expressed only through relationships.
    \item Events and Observations are the only time-indexed entities.
    \item No relationship may violate declared kind constraints.
\end{itemize}

These invariants ensure that CAE remains internally consistent and
supports formal reasoning about accountability relationships.
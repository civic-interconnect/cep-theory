% !TeX root = 00_cep_semantics.tex

\section{Jurisdictional Adapters as Oplax Functors}
\label{sec:adapters}

\subsection{Motivation}

Jurisdictions frequently maintain their own data schemas, codes, and
structural conventions.
Let $\mathbf{J_{\mathrm{local}}}$ denote the category generated by a
jurisdiction's native schema and update rules,
and let $\mathbf{J_{\mathrm{global}}}$ denote the category induced by CEP
vocabularies and canonical record structures.

An adapter must relate these two perspectives.
In practice, the mapping from local to global structure is often \emph{weak}:
optional fields may be absent locally, locally grouped elements may correspond to several
globally distinct components, and some structural detail may not be recoverable.
These characteristics motivate treating adapters as
\emph{oplax functors}, which model structure-preserving translations that
allow controlled weakening.

% ----------------------------------------------------------------------
\subsection{Oplax Functor
  \texorpdfstring{$\mathcal{A}$}{A}}
% ----------------------------------------------------------------------

A jurisdictional adapter is modeled as an oplax functor
\begin{equation}
  \mathcal{A} :
  \mathbf{J_{\mathrm{local}}}
  \longrightarrow
  \mathbf{J_{\mathrm{global}}}.
\end{equation}

For any composable morphisms
$f : X \to Y$ and $g : Y \to Z$ in
$\mathbf{J_{\mathrm{local}}}$, oplaxity provides a coherence morphism
\begin{equation}
  \phi_{f,g} :
  \mathcal{A}(g) \circ \mathcal{A}(f)
  \;\Longrightarrow\;
  \mathcal{A}(g \circ f),
\end{equation}
which need not be invertible.
This captures the possibility that two distinct local transformations
may correspond to a single global transformation,
or that information preserved locally becomes conflated at the global level.

\begin{figure}[ht]
  \centering
  \begin{tikzpicture}[node distance=3.2cm,>=Stealth]
    \node (X) {$X$};
    \node (Y) [right of=X] {$Y$};
    \node (Z) [right of=Y] {$Z$};

    \draw[->] (X) -- node[above] {$f$} (Y);
    \draw[->] (Y) -- node[above] {$g$} (Z);

    \node (AX) [below of=X, yshift=-0.2cm] {$\mathcal{A}(X)$};
    \node (AY) [below of=Y, yshift=-0.2cm] {$\mathcal{A}(Y)$};
    \node (AZ) [below of=Z, yshift=-0.2cm] {$\mathcal{A}(Z)$};

    \draw[->] (AX) -- node[above] {$\mathcal{A}(f)$} (AY);
    \draw[->] (AY) -- node[above] {$\mathcal{A}(g)$} (AZ);

    \draw[->, dashed, bend left=35]
    (AX) to node[below] {$\mathcal{A}(g \circ f)$} (AZ);

    \draw[->, shorten >=6pt, shorten <=6pt]
    ($(AX) + (0.45cm,-0.55)$) -- node[below] {$\phi_{f,g}$}
    ($(AZ) + (-0.45cm,-0.55)$);
  \end{tikzpicture}
  \caption{An oplax adapter: composition is preserved up to a coherence morphism $\phi_{f,g}$.}
  \label{fig:oplax-adapter}
\end{figure}

This formalism expresses \textbf{jurisdictional autonomy}: local schemas
may preserve distinctions or collapse details differently than the global
schema while still participating coherently in the CEP ecosystem.

\FigureCallout{Adapters as Oplax Functors (Jurisdictional Autonomy)}{
  An oplax functor represents the fact that translation from a local
  schema to the global CEP vocabulary may weaken structure.
  Local compositions $g \circ f$ may correspond to their global images only up
  to a coherence morphism $\phi_{f,g}$, allowing partial or
  jurisdiction-specific mappings while maintaining global consistency.
}

% ----------------------------------------------------------------------
\subsection{Correctness Criteria for Adapters}
% ----------------------------------------------------------------------

A jurisdictional adapter $\mathcal{A}$ is valid precisely when it
satisfies the following criteria:

\begin{enumerate}
  \item \textbf{Required-field preservation.}
        Every required field or transformation in
        $\mathbf{J_{\mathrm{local}}}$ must map to a required global term
        or transformation in $\mathbf{J_{\mathrm{global}}}$.

  \item \textbf{Optional weakening via coherence.}
        Optional or partially represented structures must be mediated by
        the oplax coherence morphisms $\phi_{f,g}$ so that weakened
        structure remains well typed in the global system.

  \item \textbf{Canonicalization compatibility.}
        The canonical identifier must be computable on all translated records:
        \begin{equation}
          \mathcal{C}(\mathcal{A}(R))
          \quad\text{is defined for all admissible local records } R.
        \end{equation}
        This ensures that each jurisdiction can produce stable and
        globally comparable SNFEI identifiers.

  \item \textbf{Provenance monotonicity.}
        Adapter-induced transformations must remain consistent with CEP's
        provenance envelopes, attestation rules, and revision ordering.

  \item \textbf{Functoriality on valid updates.}
        For every valid local update $u : R \to R'$, the induced
        $\mathcal{A}(u)$ must constitute a valid update in the global CEP
        system, modulo the necessary oplax coherence.
\end{enumerate}

A valid adapter thus functions as a structure-respecting mediator between
local civic systems and the global CEP representation, providing the
flexibility required for jurisdictional independence while maintaining
global interoperability and identifier stability.

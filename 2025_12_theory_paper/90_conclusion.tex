% !TeX root = 00_cep_semantics.tex
\section{Conclusion}

We presented a categorical semantics for the Civic Exchange Protocol
that unifies canonicalization, provenance, adapters, and context tags
into a coherent mathematical framework.
This perspective makes explicit the invariants that govern identity,
record evolution, and interoperability across heterogeneous civic data systems.

Canonicalization was formulated as a deterministic monoidal functor,
ensuring stable identifiers and well-defined equivalence classes.
Jurisdictional adapters were modeled as oplax functors, capturing how
local structure may be weakened while preserving global identity.
Context tags were expressed via a fibered category, isolating
interpretive annotations from canonical record content.
Together, these structures yield formal guarantees for identifier preservation,
compositional provenance, and cross-jurisdiction reconciliation.

The resulting semantics provides a rigorous foundation for validation,
verification, and future extensions of CEP.
It enables principled design of domain schemas, vocabularies,
and interoperability standards, while remaining extensible to evolving civic workflows.
As civic data ecosystems continue to grow in scale and complexity,
categorical methods offer a durable and expressive language for ensuring that shared
identities and exchanges remain consistent, transparent, and reliable.


\section*{Acknowledgements}

Generative AI tools were used to assist with editing and formatting during manuscript preparation.
All ideas, analyses, and conclusions are the authors' own.
All AI-assisted edits were reviewed and validated,
and the authors take full responsibility for the accuracy and integrity of this manuscript.

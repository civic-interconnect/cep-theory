% !TeX root = 00_cep_semantics.tex
\section{Introduction}

Civic information ecosystems contain heterogeneous data about entities,
relationships, and exchanges.
These data are typically fragmented across jurisdictions, systems,
formats, and organizational boundaries.
The Civic Exchange Protocol (CEP) provides a unified, verifiable,
schema-based framework for representing such data in an interoperable manner.
CEP specifies four record families: entities,
relationships, exchanges, and context tags, all wrapped by a shared
record envelope with stable identifiers, attestations, and lifecycle
metadata.

While CEP has a concrete implementation in JSON Schema, Rust, and
Python, its underlying structure is fundamentally compositional.
This paper develops a \emph{categorical semantics} for CEP.
Our
motivation is threefold: (1) to formalize the invariants that CEP
relies on for interoperability and identity stability; (2) to express
its record structures in a precise mathematical language; and (3) to
establish a foundation for evaluating transformations, adapters, and
identifier derivations across heterogeneous civic data sources.

We treat CEP canonicalization as a structured rewriting system,
which can be viewed categorically as a 2\=/category or polygraph whose
1\=/morphisms are rewrite steps and whose 2\=/morphisms capture coherence between
rewrite paths.
As in standard rewriting frameworks, not all rewrite rules
commute: information-preserving rewrites must precede information-reducing
ones.
CEP therefore adopts an ordered, stratified rewrite strategy to ensure
determinism and semantic fidelity.
This perspective is developed more fully in
Section~\ref{sec:canonicalization}.

CEP applies rewriting-theoretic ideas beyond name canonicalization,
including adapter chains, entity-merge logic, canonical-string
construction for hashing, and vocabulary evolution.
We unify several layers of rewriting:
\begin{itemize}
  \item lexical and orthographic normalization,
  \item semantic expansion (corporate forms, jurisdiction forms),
  \item schema alignment and jurisdictional adapter chains,
  \item provenance-graph normalization,
  \item vocabulary evolution and version migration,
  \item entity-merge rewriting with trust and timestamp rules,
  \item cryptographic canonicalization of canonical strings.
\end{itemize}

Conceptually, this draws on ideas from term-rewriting theory and related
areas, including compiler intermediate representations, natural-language
tokenization and normalization, linked-data graph canonicalization, blockchain determinism,
and provenance alignment~\cite{baader1998term,terese2003rewriting,spivak2014category}.

Given the heterogeneity and path-dependence of civic data pipelines, the global
rewrite system is \emph{not} designed to be:
\begin{itemize}
  \item commutative or order-independent,
  \item symmetric,
  \item context-free.
\end{itemize}
Instead, CEP is engineered as a \emph{strategy-governed} rewrite system whose
evaluation order is explicitly specified, versioned, and enforced.
Within this
framework, CEP aims for:
\begin{itemize}
  \item termination (rewriting pipelines always complete),
  \item determinism (a given input and strategy yield a unique output),
  \item consistency (strategies respect CEP invariants), and
  \item documentation (strategies and rules are inspectable and auditable).
\end{itemize}

This paper makes three main contributions:
\begin{enumerate}
  \item We formalize the Civic Exchange Protocol as a category $\mathbf{CEP}$
        whose objects are well-typed record states and whose
        morphisms represent valid, provenance-preserving transformations between them.

  \item We introduce \emph{CEP canonicalization} as a structured rewriting
        system: a stratified, strategy-governed rewriting pipeline that produces
        deterministic canonical forms suitable for hashing, comparison, and linkage.
        Categorically, we characterize canonicalization as a monoidal
        functor that induces stable identifiers (SNFEI) and preserves
        equivalence classes of civic actors and events.

  \item We give a categorical semantics for jurisdictional adapters, vocabularies,
        and envelopes: adapters are modeled as (op)lax functors mediating
        between local schema categories and the global CEP category; envelopes
        and context tags arise as natural transformations between record
        functors.
        We demonstrate that this framework extends uniformly across
        domains (campaign finance, environmental regulation, education), each
        equipped with its own controlled vocabularies and adapters, showing
        that a single compositional rewriting framework can support
        heterogeneous civic data while preserving determinism and provenance.
\end{enumerate}

The goal is not to introduce new implementation mechanisms, but to
clarify the mathematical structure implicit in CEP and justify its
identity, integrity, and interoperability guarantees.
A categorical perspective exposes CEP's compositionality, makes explicit the
conditions under which identifiers and transformations remain stable,
and provides a principled basis for future extensions in federation,
data fusion, and cross-jurisdiction civic analytics.
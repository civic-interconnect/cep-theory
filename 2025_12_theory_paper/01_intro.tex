% ============================================================
\section{Introduction}
% ============================================================

Civic information ecosystems contain heterogeneous data about
entities, relationships, and exchanges. These data are typically
fragmented across jurisdictions, systems, formats, and organizational
boundaries. The Civic Exchange Protocol (CEP) provides a unified,
verifiable, schema-based framework for representing such data in an
interoperable manner. CEP specifies four record families---entities,
relationships, exchanges, and context tags---all wrapped by a shared
record envelope with stable identifiers, attestations, and lifecycle
metadata.

While CEP has a concrete implementation in JSON Schema, Rust, and
Python, its underlying structure is fundamentally compositional.
This paper develops a \emph{categorical semantics} for CEP. Our
motivation is threefold: (1) to formalize the invariants that CEP
relies on for interoperability and identity stability; (2) to express
its record structures in a precise mathematical language; and (3) to
establish a foundation for evaluating transformations, adapters, and
identifier derivations across heterogeneous civic data sources.

Our contributions are:
\begin{enumerate}
    \item A construction of a category \textbf{CEP} whose objects are
    well-typed record states and whose morphisms represent valid,
    provenance-preserving transformations.
    \item A functorial account of envelopes, attestations, and context
    tags as natural transformations between appropriate record
    functors.
    \item A characterization of canonicalization as a monoidal
    functor that induces stable identifiers (SNFEI) and preserves
    equivalence classes of civic actors and events.
    \item A semantics for jurisdictional adapters as oplax functors
    mediating between local schemas and global vocabularies.
\end{enumerate}

The goal is not to introduce new implementation mechanisms, but to
clarify the mathematical structure implicit in CEP and justify its
identity, integrity, and interoperability guarantees. A categorical
perspective exposes CEP's compositionality, makes explicit the
conditions under which identifiers and transformations remain stable,
and provides a principled basis for future extensions in federation,
data fusion, and cross-jurisdiction civic analytics.

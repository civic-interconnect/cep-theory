% ============================================================
\section{Applications}
% ============================================================
\label{sec:apps}

This section illustrates how the categorical semantics developed above
manifest in real-world civic data workflows. Each example highlights a
core CEP invariant and shows how the categorical structure enforces it.

\subsection{Civic Entity Records: Identity Stability}

A municipal entity’s lifecycle is modeled as a chain of morphisms in
$\mathbf{CEP}$:
\[
E_1 \xrightarrow{f_1} E_2 \xrightarrow{f_2} \cdots
\]
where each $E_i$ is a revision of the same entity.
Identity Invariance (Section~3.2) ensures that any
identity-preserving amendment $f_i$ leaves the canonical form unchanged:
\[
\mathcal{C}(E_i) = \mathcal{C}(E_{i+1}).
\]
Thus the SNFEI remains stable even as auxiliary data (e.g., address,
classification, or contact fields) evolves over time.

\subsection{Campaign Finance: Compositional Provenance}

A donation from a donor $D$ to a committee $C$ is modeled as an exchange
morphism $f : D \to C$. A subsequent transfer from $C$ to a subcommittee
$S$ is a morphism $g : C \to S$. The composed morphism
$g \circ f : D \to S$ represents the derived provenance lineage.
Because composition in $\mathbf{CEP}$ is associative, this lineage is
canonical and immutable, enabling consistent financial traceability
across multiple reporting systems.

\subsection{Public Contracting: Data Fusion via Pullbacks}

Consider a Vendor Registry record $R_V$ and a Contract Record $R_C$
that both reference the same underlying legal entity $A$.
The pullback
\[
P = R_V \times_A R_C
\]
constructs the maximal consistent set of facts compatible with both
records. This is exactly the categorical mechanism that guarantees
sound cross-system data integration (Section~3.3) and supports
audits that reconcile public procurement data across heterogeneous
jurisdictional sources.

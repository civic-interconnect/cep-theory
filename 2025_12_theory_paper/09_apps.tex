% !TeX root = 00_cep_semantics.tex

\section{Applications}
\label{sec:apps}

The categorical semantics developed above manifest directly in real civic
data workflows.
Domain schemas provide typed structures for entities and exchanges,
while CEP vocabularies furnish the controlled classifications
and relationship types used throughout these workflows.
The following examples illustrate how functoriality, monoidality, pullbacks,
and oplax adapter semantics enforce CEP's core invariants in practice.

% ------------------------------------------------------------
\subsection{Civic Entity Records: Identity Stability}
% ------------------------------------------------------------

Every domain schema (e.g., municipal, educational, environmental) defines
a category of entity revisions whose morphisms correspond to admissible
updates.
A municipal entity's lifecycle is therefore a chain in
$\mathbf{CEP}$:
\[
  E_1 \xrightarrow{f_1} E_2 \xrightarrow{f_2} \cdots
\]
where each $E_i$ conforms to the relevant domain schema and vocabulary.

Identity Invariance ensures that any identity-preserving update $f_i$
leaves the canonical form unchanged:
\[
  \mathcal{C}(E_i) = \mathcal{C}(E_{i+1}).
\]
Thus the SNFEI remains stable even as auxiliary fields (address,
classification, jurisdictional codes, vocabulary-driven status fields)
evolve over time.
The vocabulary layer guarantees that enumerated fields
(e.g.\ \texttt{ACTIVE}, \texttt{DISSOLVED}) remain consistent across
revisions and jurisdictions.

This yields a strong operational principle:

\begin{quote}
  \textbf{Domain schemas define the structure of change; canonicalization
    guarantees that such change never affects identity.}
\end{quote}

% ------------------------------------------------------------
\subsection{Campaign Finance: Compositional Provenance}
% ------------------------------------------------------------

Domain schemas for campaign finance define typed exchange morphisms,
where controlled vocabulary terms specify the semantics of each transfer
(e.g.\ \texttt{donation}, \texttt{in-kind}, \texttt{reallocation}).

A donation from a donor $D$ to a committee $C$ is a morphism
$f : D \to C$.
A downstream transfer from $C$ to a subcommittee $S$ is
a morphism $g : C \to S$.
Their composition
\[
  g \circ f : D \to S
\]
represents the derived provenance lineage.

Because composition in $\mathbf{CEP}$ is associative and is compatible with the vocabulary-governed domain schema,
this lineage is canonical across all reporting systems.
It enables coherent financial traceability even
when jurisdictions differ in local reporting conventions—oplax adapters
(Section~\ref{sec:adapters}) ensure that structure-weakening translations
still preserve this compositional provenance.

% ------------------------------------------------------------
\subsection{Public Contracting: Data Fusion via Pullbacks}
% ------------------------------------------------------------

Public contracting data arises from multiple domain schemas:
\begin{itemize}
  \item Vendor Registry schema (legal entities, certifications, status),
  \item Contract schema (awards, amendments, payments),
  \item Jurisdiction-specific extensions and vocabularies.
\end{itemize}

Suppose a Vendor Registry record $R_V$ and a Contract Record $R_C$ both
reference the same underlying legal entity $A$.
The fiber product
\[
  P = R_V \times_A R_C
\]
constructs the maximal consistent set of jointly satisfiable facts.
This pullback operation merges information along the shared canonical identity
specified by the SNFEI and vocabulary-controlled relationship types.

This is the categorical mechanism that guarantees sound cross-system
integration.
It ensures that:

\begin{itemize}
  \item canonical identity aligns disparate domain schemas,
  \item vocabulary-governed code systems reconcile classification fields,
  \item provenance constraints remain intact across fused datasets.
\end{itemize}

As a result, auditors can reconcile procurement data across heterogeneous
jurisdictional sources, even when local schemas differ substantially, while
maintaining CEP's global invariants.


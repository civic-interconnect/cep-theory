% ============================================================
\section{Context Tags as Indexed Families}
\label{sec:ctags}
% ============================================================

\subsection{\texorpdfstring{Fibered Category
        $\pi : \mathbf{CT} \to \mathbf{CEP}$}
    {Fibered Category pi : CT → CEP}}
% ------------------------------------------------------------

Context tags (CTags) are interpretive annotations attached to a record
without altering its canonical identity. Their semantics is most
naturally expressed via a \emph{fibered category}:
\begin{equation}
    \pi : \mathbf{CT} \longrightarrow \mathbf{CEP},
\end{equation}
where:
\begin{itemize}
    \item $\mathbf{CEP}$ is the base category of identity-bearing records,
    \item $\mathbf{CT}$ is the category whose objects are
          \emph{pairs} $(R, T)$ of a record and a context tag attached to it,
    \item $\pi$ is the projection functor $\pi(R, T) = R$.
\end{itemize}

For each base object $R \in \mathbf{CEP}$, the \emph{fiber}
$\mathbf{CT}_R$ is the collection of all valid context tags that may be
attached to $R$. These fibers encode interpretive or analytic
information without modifying the canonical structure of the underlying
record.

\FigureCallout{Context Tags in a Fibered Category (Separation of Concerns)}{
    The fibration $\pi : \mathbf{CT} \to \mathbf{CEP}$ cleanly separates
    identity (base objects in $\mathbf{CEP}$) from interpretation (fibers
    $\mathbf{CT}_R$). Any morphism in the base category induces a structured
    reindexing between fibers, guaranteeing that context tags track valid
    record evolutions without affecting the canonical identifier. This
    formalizes the principle that CTags annotate a record but never influence
    its SNFEI identity.
}

% ------------------------------------------------------------
\subsection{Functoriality and Reindexing}
% ------------------------------------------------------------

Given a morphism
\begin{equation}
    f : R \longrightarrow R'
\end{equation}
in the base category $\mathbf{CEP}$ (representing a valid record
evolution or update), the fibration supplies a corresponding
\emph{reindexing functor}:
\begin{equation}
    f^\ast : \mathbf{CT}_{R'} \longrightarrow \mathbf{CT}_R.
\end{equation}

Intuitively, $f^\ast$ describes how tags on a revised record $R'$ may be
pulled back to valid tags on the earlier state $R$, preserving their
interpretive meaning.

\begin{figure}[h]
    \centering
    \begin{tikzpicture}[node distance=2.8cm,>=Stealth]
        \node (CR) {$\mathbf{CT}_R$};
        \node (CRp) [right of=CR] {$\mathbf{CT}_{R'}$};

        \node (R) [below of=CR] {$R$};
        \node (Rp) [below of=CRp] {$R'$};

        \draw[->] (CRp) -- node[above] {$f^\ast$} (CR);
        \draw[->] (R) -- node[below] {$f$} (Rp);

        \draw[->] (CR) -- node[left] {$\pi$} (R);
        \draw[->] (CRp) -- node[right] {$\pi$} (Rp);
    \end{tikzpicture}
    \caption{Reindexing in the fibration $\pi : \mathbf{CT} \to \mathbf{CEP}$.}
    \label{fig:fibration-reindex}
\end{figure}

The reindexing condition ensures:
\begin{equation}
    \pi \circ f^\ast = f \circ \pi,
\end{equation}
i.e., the diagram in Figure~\ref{fig:fibration-reindex} commutes.

This formalizes an essential CEP design principle:

\begin{quote}
    \textbf{Context tags move with the record's evolution but do not change
        its identity.}
\end{quote}

% ------------------------------------------------------------
\subsection{Identity Preservation and Canonical Invariance}
% ------------------------------------------------------------

A tag object $T \in \mathbf{CT}_R$ must not alter any canonical data
used in identifier generation. Formally:

\begin{equation}
    \mathcal{C}(R) = \mathcal{C}(R,T),
\end{equation}

where $\mathcal{C}$ is the canonicalization functor defined previously.
Equivalently, the projection functor $\pi$ is \emph{identity-reflecting}:
if $(R,T)$ and $(R',T')$ share the same canonical identifier, then
$R$ and $R'$ must already do so in $\mathbf{CEP}$.

Thus:
\begin{itemize}
    \item CTags are \emph{pure annotations};
    \item they cannot introduce new canonical structure;
    \item they cannot affect the SNFEI or equivalence classes of records.
\end{itemize}

This guarantees a strict separation between:
\[
    \text{identity} \quad\text{(base category)} \qquad\text{and}\qquad
    \text{interpretation} \quad\text{(fibers)}.
\]


% ============================================================
% Appendix E: Glossary for Non-Category Theory Readers
% ============================================================
\clearpage
\section*{Appendix E. Glossary for Non-Category Theory Readers}
\addcontentsline{toc}{section}{Appendix E. Glossary for Non-Category Theory Readers}
This appendix provides short, non-technical explanations of the categorical
concepts used in the paper.  The intention is to make the mathematical structure
of CEP more accessible to readers from computer science, data engineering,
public administration, and civic-technology communities.

% ------------------------------------------------------------
\subsection*{E.1 Categories}
A \emph{category} is a mathematical setting that describes:
\begin{itemize}
    \item some \textbf{objects} (things), and
    \item some \textbf{morphisms} (arrows) between them.
\end{itemize}

You can think of a category as a directed graph with rules:
every morphism has a source and target, arrows can be composed, and each object
has an identity arrow that does nothing.

In CEP:
\begin{itemize}
    \item objects = record states,
    \item morphisms = valid record transformations (updates, amendments, joins).
\end{itemize}

% ------------------------------------------------------------
\subsection*{E.2 Functors}
A \emph{functor} is a mapping between categories that preserves structure.
It sends:
\begin{itemize}
    \item each object to another object, and
    \item each morphism to another morphism,
\end{itemize}
in a way that respects composition and identity.

In CEP, a functor often represents a pipeline stage, such as:
\begin{itemize}
    \item wrapping a payload in an envelope,
    \item normalizing noisy text into canonical components,
    \item assembling canonical strings.
\end{itemize}

Functors ensure that if a record evolves legally, its transformed
version evolves legally too.

% ------------------------------------------------------------
\subsection*{E.3 Natural Transformations}
A \emph{natural transformation} is a structured way of comparing two functors.
If functors are "processing stages," a natural transformation is a
systematic way to convert the output of one stage into the output of another.

In CEP, attestations are modeled as natural transformations:
\begin{itemize}
    \item the envelope functor produces a plain metadata wrapper,
    \item the attested-envelope functor produces a cryptographically validated wrapper.
\end{itemize}

Naturality expresses the idea:
\begin{quote}
    "Whether you process then attest, or attest then process,
    you end up with consistent provenance."
\end{quote}

% ------------------------------------------------------------
\subsection*{E.4 Monoidal Categories}
A \emph{monoidal category} is a category equipped with a notion of
"combining things."

Examples:
\begin{itemize}
    \item strings combine by concatenation,
    \item datasets combine by joining,
    \item workflows combine by sequencing.
\end{itemize}

Canonicalization in CEP is monoidal because it combines
individual normalized components into a single canonical string
in a strictly deterministic order.

% ------------------------------------------------------------
\subsection*{E.5 Strict Monoidal Functors}
A \emph{strict monoidal functor} is a functor that preserves the
combination structure exactly.

In CEP:
\begin{itemize}
    \item the order of pieces (name, address, date, jurisdiction)
          must always be preserved,
    \item no additional symbols or whitespace are introduced,
    \item the final output is the canonical string fed to SHA--256.
\end{itemize}

This strictness is what guarantees the stability of the SNFEI identifier.

% ------------------------------------------------------------
\subsection*{E.6 Oplax Functors}
An \emph{oplax functor} preserves structure in a weakened way.

Think of it as "oppositionally lax".  
It is "lax" as in loose, but the op- signals that the direction of the non-invertible natural transformation ($\alpha$) is reversed or "opposite".

It allows:
\begin{itemize}
    \item missing fields,
    \item lossy interpretations,
    \item mappings that preserve meaning but not full structure.
\end{itemize}

Jurisdictional adapters in CEP behave oplaxly because:
\begin{itemize}
    \item local data models may omit fields,
    \item global vocabularies may have stricter typing,
    \item some equivalences hold only "up to" a coherence rule.
\end{itemize}

Oplax behavior models "local autonomy with global convergence."

% ------------------------------------------------------------
\subsection*{E.7 Pullbacks (Consistent Joins)}
A \emph{pullback} is the categorical notion of a consistent join.

If two data sources both refer to the same entity or event,
the pullback constructs the most precise version of their agreement.

This formalizes CEP's guarantee that:
\begin{quote}
    Records may be joined only when they assert compatible facts.
\end{quote}

% ------------------------------------------------------------
\subsection*{E.8 Fibered Categories}
A \emph{fibered category} describes a setting where each object has a
family of additional structures "above it."

In CEP:
\begin{itemize}
    \item the base category is $\mathbf{CEP}$,
    \item the fibers contain context tags (CTags).
\end{itemize}

This cleanly separates:
\begin{itemize}
    \item \textbf{identity} (in the base category), and
    \item \textbf{interpretation or annotation} (in the fiber).
\end{itemize}

This is why CTags do not affect the SNFEI identifier.

% ------------------------------------------------------------
\subsection*{E.9 Universal Properties}
A \emph{universal property} describes an object that is "best" or "most
canonical" for a specific purpose.

SNFEI behaves like a universal construction because:

\begin{itemize}
    \item it is determined by a canonical string,
    \item it is invariant under allowed morphisms,
    \item any other identifier consistent with CEP's invariants must
          factor uniquely through this construction.
\end{itemize}

This is the mathematical justification for treating SNFEI as a stable,
verifiable, compositional global identifier.

% ------------------------------------------------------------
\subsection*{E.10 Summary Table}
\begin{center}
    \begin{tabular}{p{0.20\linewidth} p{0.35\linewidth} p{0.35\linewidth}}
        \toprule
        \textbf{Concept}        & \textbf{Intuition}         & \textbf{CEP Role}           \\
        \midrule
        Category                & Things, allowed changes & Record states and updates   \\
        Functor                 & Structure-preserving map   & Normalization, envelopes \\
        Natural trans-formation & Coherent comparison        & Attestations                \\
        Monoidal category       & Combine things             & Canonical assembly          \\
        Strict monoidal functor & Combine exactly            & SNFEI stability             \\
        Oplax functor           & Weak structure map         & Jurisdiction adapters       \\
        Pullback                & Consistent join            & Merging record fragments    \\
        Fibered category        & Per-object annotations     & CTags above records         \\
        Universal property      & Optimal construction       & Identifier uniqueness       \\
        \bottomrule
    \end{tabular}
\end{center}

% ------------------------------------------------------------
\subsection*{E.11 Closing Note}
These notions are not introduced for abstraction's sake—they express
precisely the structural guarantees CEP requires for interoperability,
trust, and cross-jurisdiction governance.

They allow the protocol to be:
\begin{itemize}
    \item modular,
    \item formally verifiable,
    \item robust to heterogeneity,
    \item and extensible to future domains.
\end{itemize}

By grounding CEP in category theory, we provide a rigorous foundation
for its design principles and operational claims.

% !TeX root = 00_cep_semantics.tex
\clearpage
\section*{Appendix E. Glossary for Non-Category Theory Readers}
\addcontentsline{toc}{section}{Appendix E. Glossary for Non-Category Theory Readers}

This appendix provides short, non-technical explanations of the categorical
concepts used in the paper.
The intention is to make the mathematical structure
of CEP more accessible to readers from computer science, data engineering,
public administration, and civic-technology communities.

% ------------------------------------------------------------
\subsection*{E.1 Categories}
A \emph{category} is a mathematical setting that describes:
\begin{itemize}
  \item some \textbf{objects} (things), and
  \item some \textbf{morphisms} (arrows) between them.
\end{itemize}

A category is like a directed graph with rules:
every arrow (morphism) has a source and target,
arrows can be composed,
and each object has an identity arrow that starts from the object, points back to the object, and does nothing.

The latin root \emph{morph} means "form" or "shape",
so a morphism is a way of changing the form of one object into another.

In CEP:
\begin{itemize}
  \item objects = record states,
  \item morphisms = valid record transformations (updates, amendments, joins).
\end{itemize}

Defining CEP as a category formalizes the idea of "things and the allowed changes between them".

% ------------------------------------------------------------
\subsection*{E.2 Functors}
A \emph{functor} is a mapping between categories that preserves structure.
It sends:
\begin{itemize}
  \item each object to another object, and
  \item each morphism to another morphism,
\end{itemize}
in a way that respects composition and identity.

In CEP, a functor often represents a pipeline stage, such as:
\begin{itemize}
  \item wrapping a payload in an envelope,
  \item normalizing noisy text into canonical components,
  \item assembling canonical strings.
\end{itemize}

Functors ensure that if a record evolves legally, its transformed
version evolves legally too.

\emph{Function} and \emph{functor} have the same root and similarities,
but different meanings:
functions map elements within sets,
while functors map objects and morphisms between categories.

% ------------------------------------------------------------
\subsection*{E.3 Natural Transformations}
A \emph{natural transformation} is a structured way of comparing two functors.
If functors are "processing stages", a natural transformation is a
systematic way to convert the output of one stage into the output of another.

In CEP, attestations are modeled as natural transformations:
\begin{itemize}
  \item the envelope functor produces a plain metadata wrapper,
  \item the attested-envelope functor produces a cryptographically validated wrapper.
\end{itemize}

Naturality expresses the idea:
\begin{quote}
  "Whether you process then attest, or attest then process,
  you end up with consistent provenance."
\end{quote}

% ------------------------------------------------------------
\subsection*{E.4 Monoidal Categories}
A \emph{monoidal category} is a category equipped with a notion of
"combining things".

Examples:
\begin{itemize}
  \item strings combine by concatenation,
  \item datasets combine by joining,
  \item workflows combine by sequencing.
\end{itemize}

Canonicalization is the process of converting data into a standard format,
most commonly by selecting a single, preferred output
to represent a piece of content that could have multiple versions.

Canonicalization in CEP is \emph{monoidal} because it combines
individual normalized components into a single canonical string
in a strictly deterministic order.

% ------------------------------------------------------------
\subsection*{E.5 Strict Monoidal Functors}
A \emph{strict monoidal functor} is a functor that preserves the
combination structure \emph{exactly}.

In CEP:
\begin{itemize}
  \item the order of pieces (name, address, date, jurisdiction)
        must always be preserved,
  \item no additional symbols or whitespace are introduced,
  \item the final output is the canonical string fed to the SHA-256 cryptographic hash function.
\end{itemize}

This strictness is what guarantees the stability of the SNFEI identifier.

% ------------------------------------------------------------
\subsection*{E.6 Oplax Functors}
An \emph{oplax functor} preserves structure in a weakened, direction-sensitive way.
The \emph{op} prefix indicates \emph{opposite} directionality and \emph{lax} indicates looseness.
An oplax functor therefore "loosens" structure in a specific direction and
allows "preservation up to a coherence map" rather than strict equality.
A coherence map is a controlled way of relating two structures that are not strictly equal.

It allows:
\begin{itemize}
  \item missing fields,
  \item lossy interpretations,
  \item mappings that preserve meaning but not full structure.
\end{itemize}

Jurisdictional adapters in CEP behave oplaxly because:
\begin{itemize}
  \item local data models may omit fields,
  \item global vocabularies may have stricter typing,
  \item some equivalences hold only "up to" a coherence rule.
\end{itemize}

Oplax behavior models "local autonomy with global convergence".
It means that local jurisdictions can adapt data flexibly
while still ensuring that the global system remains coherent and consistent.

% ------------------------------------------------------------
\subsection*{E.7 Pullbacks (Consistent Joins)}
A \emph{pullback} is the categorical notion of a \emph{consistent join}.

If two data sources both refer to the \emph{same entity or event},
the pullback constructs the most precise version of their agreement.

This formalizes CEP's guarantee that:
\begin{quote}
  Records may be joined only when they assert compatible facts.
\end{quote}

A pullback enables us to formally define the data fusion operation that
combines records from different schemas while preserving:
\begin{itemize}
  \item canonical identity,
  \item vocabulary-governed semantics,
  \item provenance constraints.
\end{itemize}

\emph{Data fusion} refers to the process of integrating multiple data sources
to produce more consistent, accurate, and useful information than that
provided by any individual source.

% ------------------------------------------------------------
\subsection*{E.8 Fibered Categories}
A \emph{fibered category} describes a setting where each object has a
family of additional structures "above it".

CEP is a fibered category.
In CEP:
\begin{itemize}
  \item the base category is $\mathbf{CEP}$,
  \item the fibers contain context tags (CTags).
\end{itemize}

This cleanly separates:
\begin{itemize}
  \item \textbf{identity} (in the base category), and
  \item \textbf{interpretation or annotation} (in the fiber).
\end{itemize}

CTags provide information about a record, without affecting the canonical SNFEI identifier.

% ------------------------------------------------------------
\subsection*{E.9 Universal Properties}
A \emph{universal property} describes an object that is "best" or "most canonical" for a specific purpose.

SNFEI behaves like a universal property construction because:

\begin{itemize}
  \item it is determined by a canonical string,
  \item it is invariant under allowed morphisms,
  \item any other identifier consistent with CEP's invariants must factor uniquely through this construction.
\end{itemize}

This is the mathematical justification enabling us to treat SNFEI
as a stable, verifiable, compositional global identifier.

  % ------------------------------------------------------------
  {\small
    \subsection*{E.10 Summary Table}
    \begin{center}
      \begin{tabular}{p{0.23\linewidth} p{0.35\linewidth} p{0.32\linewidth}}
        \toprule
        \textbf{Concept}        & \textbf{Intuition}       & \textbf{CEP Role}         \\
        \midrule
        Category                & Things, allowed changes  & Record states and updates \\
        Functor                 & Structure-preserving map & Normalization, envelopes  \\
        Natural transformation  & Coherent comparison      & Attestations              \\
        Monoidal category       & Combine things           & Canonical assembly        \\
        Strict monoidal functor & Combine exactly          & SNFEI stability           \\
        Oplax functor           & Weak structure map       & Jurisdiction adapters     \\
        Pullback                & Consistent join          & Merging record fragments  \\
        Fibered category        & Object annotations       & CTags above records       \\
        Universal property      & Optimal construction     & Identifier uniqueness     \\
        \bottomrule
      \end{tabular}
    \end{center}
  }

% ------------------------------------------------------------
\subsection*{E.11 Closing Note}
These notions are not introduced for abstraction's sake;
they express precisely and formally the structural guarantees
CEP requires to support interoperability, trust, and cross-jurisdiction governance.

These mathematical definitions allow the protocol to be:
\begin{itemize}
  \item modular in its design,
  \item formally verifiable in its behavior,
  \item capable of operating consistently even when sources use different schemas, vocabularies, or jurisdictional rules,
  \item and extensible to future domains.
\end{itemize}

By grounding CEP in category theory,
we provide a rigorous foundation for its design principles and operational claims.

% !TeX root = 00_cep_semantics.tex
\section{Limitations and Future Work}
\label{sec:limitations}

The categorical core presented in this paper captures identity,
canonicalization, provenance, and interoperability for discrete,
revision-based civic records.
Several important extensions remain outside the present scope.

Many civic systems also generate data that is statistical, uncertain, or
continuously updated (e.g., longitudinal indicators, evolving aggregates).
Incorporating such information may require probabilistic semantics or
temporal indexing, which are not modeled in the current framework.

Jurisdictions evolve data models over time.
Although CEP supports adapters for structural variation,
a full account of schema evolution, including additions,
deprecations, and long-term migration paths, remains future work.

\paragraph{Multi-Stage and Nested Exchanges.}
Some workflows involve layered or multi-party processes
(e.g., multi-level budget allocations, nested reporting pipelines).
These may benefit from higher-structured categorical tools,
but such extensions lie beyond the scope of this foundational treatment.

\paragraph{Rule Sensitivity in Canonicalization.}
Certain linguistic cases (such as expansions of abbreviations like “S.A.”)
require stratified rule ordering within the normalization pipeline rather
than treating all rewrite rules as freely permutable.
This refinement does not affect the well-definedness of the canonicalization
function but does highlight the need for continued empirical tuning of rule strata.

\medskip
The present semantics establish a robust core for identity and interoperability.
Extending CEP to the richer data practices found across governments and civic ecosystems
is a key direction for future work.

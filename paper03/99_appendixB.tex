% !TeX root = 00P3_cee_verticals.tex
\section*{Appendix B. Worked Examples}
\label{app:B}
\addcontentsline{toc}{section}{Appendix B. Worked Examples}

This appendix presents two concrete worked examples of bicategorical
interpretation: one for SME-friendly procurement and one for community
asset access.

\subsection*{B.1 SME-Friendly Procurement}

Consider a procurement lot with noisy inputs:
multiple spellings of the supplier's name,
ambiguous CPV codes,
and missing procedure-type metadata.

\paragraph{Step 1: Canonicalization (CEP base).}
\begin{enumerate}
  \item Normalize the entity fields (legal name, jurisdiction, value).
  \item Produce a canonical string in fixed order.
  \item Compute SNFEI via SHA-256.
\end{enumerate}

\paragraph{Step 2: Adapter semantics.}
Jurisdictional quirks (missing CPV codes, inconsistent currencies)
are handled via oplax functorial rules.

\paragraph{Step 3: CEE explanation.}
Evidence: low estimated value, open procedure type, minimal documentation.
Attribution: rule-based model ``sme-rule-v1''.
Narrative: ``This lot appears SME-friendly because \dots''.

Here the explanation is a 2-morphism refining the procurement relationship.

\subsection*{B.2 Community Asset Access}

Consider a neighborhood polygon and a dataset of parks and libraries.

Step 1: Construct CEP entities:
\begin{itemize}
  \item area entity (neighborhood),
  \item asset entities (parks, libraries),
  \item relationships linking assets to areas.
\end{itemize}

Step 2: Compute evidence layers and metrics:
population-served,
distance-to-assets,
equity index.

Step 3: Perform CEE prioritization:
Based on computed metrics and attribution model,
the vertical outputs a bundle with AREA\_ACCESS\_PRIORITY.
This bundle is a 2-morphism living above the area's incoming and
outgoing relationships.

\subsection*{B.3 Composition Across Verticals}

A municipality appearing in both verticals
supports functorial maps aligning their CEP entities and enabling
interoperable explanations.


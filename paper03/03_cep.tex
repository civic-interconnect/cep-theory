% !TeX root = 00P3_cee_verticals.tex

\section{Recap of the CEP Canonical Layer}
\label{sec:cep-recap}

The first paper in this series develops CEP as a rewriting-based,
categorical framework for civic data.
Here we briefly recap the aspects that are most relevant for
vertical domains and civic explanations.

% ------------------------------------------------------------
\subsection{Canonical Entities and Relationships}
% ------------------------------------------------------------

Let $\mathcal{E}$ denote the collection of CEP entity schemas, and
$\mathcal{R}$ the collection of relationship schemas.
Each entity instance is normalized by a strategy-controlled rewriting
system that:
\begin{itemize}[nosep]
  \item canonicalizes lexical forms (names, addresses, codes);
  \item aligns fields with schema-defined structures;
  \item resolves identifiers and computes SNFEI-style fingerprints.
\end{itemize}

The result is a set of canonical entities and relationships that can be
treated as objects and morphisms in a category we denote by
$\mathbf{CEP}$.

% ------------------------------------------------------------
\subsection{Graphs, Envelopes, and Provenance}
% ------------------------------------------------------------

CEP records live not in isolation but as graphs: entities linked by
relationships, wrapped in record envelopes that carry provenance and
validation evidence.
The canonical encoding specification (CEC) and graph normalization
specification (GNS) ensure that equivalent graphs normalize to identical
representations and hashes.

For the purposes of this paper, it is enough to view:
\begin{itemize}
  \item \emph{Objects} of $\mathbf{CEP}$ as canonical entity-graph
        components (possibly with envelopes attached);
  \item \emph{Morphisms} as (typed) relationships, exchanges, or
        validated graph transformations between such components.
\end{itemize}

% ------------------------------------------------------------
\subsection{Adapters as Functorial Bridges}
% ------------------------------------------------------------

Adapters map raw data from external sources into the canonical CEP
universe.
Each adapter:
\begin{enumerate}[nosep]
  \item canonicalizes the input (lexical and semantic normalization);
  \item aligns it to CEP schemas;
  \item computes identities and attaches attestations.
\end{enumerate}

Operationally, an adapter behaves like a functor from a ``source
category'' of raw records to the target category $\mathbf{CEP}$.
In the first paper, this is treated primarily at the level of rewriting
and graph semantics; here we build on that intuition to organize
vertical domains.

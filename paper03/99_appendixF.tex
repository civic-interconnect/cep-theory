% !TeX root = 00P3_cee_verticals.tex

\clearpage
\section*{Appendix F. Supporting Categorical Terminology}
\addcontentsline{toc}{section}{Appendix F. Supporting Categorical Terminology}

This appendix defines categorical terms used in CEE that extend beyond the
foundational concepts in CEP Appendix~E. Readers unfamiliar with category
theory should consult CEP's glossary first for definitions of \emph{functor},
\emph{natural transformation}, \emph{monoidal category}, \emph{pullback},
and \emph{fibered category}.

\begin{description}[style=nextline,leftmargin=1.5cm]

  \item[Cartesianness]
        A property ensuring structured data can be ``pulled back'' along
        relationships without losing information. Cartesian structure guarantees
        that restricting a federal dataset to a state jurisdiction preserves all
        relevant relationships. See \emph{pullback} in CEP Appendix~E.

  \item[Compositionality]
        The principle that complex structures are built from simpler parts, with
        the meaning of the whole determined by the parts and how they combine.
        CEE's compositional design means evidence chains can be assembled from
        individual attestations.

  \item[Definedness]
        Whether an operation or relationship is valid for given inputs. Partial
        functions may be undefined for some inputs; CEE explicitly tracks
        definedness to distinguish ``no data'' from ``not applicable.''

  \item[Endofunction]
        A function from a set to itself. Update operations on entity state are
        endofunctions: they take a state and return a new state of the same type.

  \item[Fibered Category]
        A category equipped with a structure-preserving projection to a base
        category, allowing objects and morphisms to be indexed by elements of
        that base. In CEE, vertical domains are modeled as fibered categories over
        the CEP base category, enabling domain-specific semantics layered atop a
        shared canonical infrastructure. See \emph{fibered category} in CEP
        Appendix~E.

  \item[Fibered Structure]
        The organizational pattern induced by a fibered category, in which
        objects are grouped into fibers over a base category according to
        context or domain. In CEE, fibered structure allows vertical domains to
        specialize interpretation and evidence while remaining coherently
        anchored to the same underlying CEP entities and morphisms.

  \item[Functoriality]
        The property of preserving structure across transformations. A mapping
        is \emph{functorial} if it respects composition and identity---translating
        data between systems without corrupting accountability relationships.
        See \emph{functor} in CEP Appendix~E.

  \item[Lifting]
        Given a relationship at one level, finding a corresponding relationship
        at a higher level that projects down to it. Lifting allows inference of
        jurisdiction-level patterns from entity-level exchanges.

  \item[Monoidality]
        Structure allowing associative combination with a neutral element.
        Monetary values are monoidal under addition (zero is neutral). A mapping
        is \emph{monoidal} if it preserves this combining structure.
        See \emph{monoidal category} in CEP Appendix~E.

  \item[Naturality]
        A transformation is \emph{natural} if it works uniformly across all
        instances without depending on arbitrary choices. Natural mappings are
        canonical rather than ad hoc.
        See \emph{natural transformation} in CEP Appendix~E.

  \item[Oplaxity]
        A relaxed form of structure preservation where equations become directed
        inequalities. Oplax mappings allow ``preservation up to a coherence map''
        rather than strict equality, accommodating real-world data where
        composition may lose information.
        See \emph{oplax functor} in CEP Appendix~E.

  \item[Permutability]
        The property that certain operations can be performed in any order with
        the same result. Permutable operations simplify parallel processing and
        reduce coordination requirements in distributed systems.

  \item[Preordered]
        A set equipped with a reflexive and transitive relation (not necessarily
        antisymmetric). Time stamps and version numbers form preorders,
        generalizing ``earlier than'' or ``less detailed than.''

  \item[Quotienting]
        Collapsing distinctions by treating related elements as equivalent.
        Quotienting by jurisdiction aggregates city-level data into state-level
        summaries, treating individual cities as interchangeable for certain
        analyses.

\end{description}

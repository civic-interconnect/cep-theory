% !TeX root = 00P3_cee_verticals.tex

\section{Research Agenda}
\label{sec:agenda}

The preceding sections outline a conceptual framework.
Turning this into a mature theory and a robust body of practice
requires sustained work along several fronts.
Here we sketch a research agenda structured around three themes:
formalization, empirical validation, and tooling.

% ------------------------------------------------------------
\subsection{Formalization}
% ------------------------------------------------------------

On the formal side, several questions arise naturally:

\begin{itemize}
  \item \textbf{Bicategory structure.}
        Precisely characterize the bicategory of civic entities,
        relationships, and explanations.
        %
        Specify identity and composition laws for 2-morphisms and
        prove coherence theorems.
  \item \textbf{Vertical domains as fibres.}
        Develop a fibration or indexed-category perspective in which
        each vertical domain appears as a fibre over a base of CEP
        schemas and vocabularies.
  \item \textbf{Transport of explanations.}
        Characterize when explanations in one vertical can be
        transported to another via functors and 2-functors between
        corresponding categories.
  \item \textbf{Universal properties.}
        Investigate whether certain verticals satisfy universal
        properties (limits, colimits, adjunctions) with respect to
        others, especially in multi-criteria decision settings.
\end{itemize}

These questions are not merely abstract.
They capture when and how explanations can be safely reused, compared,
or composed across domains.

% ------------------------------------------------------------
\subsection{Empirical Validation}
% ------------------------------------------------------------

The framework must also be tested against real civic data and
realistic decision contexts.

\begin{itemize}
  \item \textbf{Case studies.}
        Develop detailed case studies for a small set of verticals
        (such as SME procurement and community asset access), tracing
        the entire pipeline from raw data through CEP, adapters, and
        CEE into concrete decisions.
  \item \textbf{Identity robustness.}
        Evaluate the stability and robustness of SNFEI identifiers and
        other identity schemes under noisy and adversarial naming
        conditions, since identity is central to cross-vertical
        reasoning.
  \item \textbf{Explanation quality.}
        Assess whether CEE-style explanations improve human
        understanding, trust, and decision quality compared to
        baseline dashboards or opaque scores.
  \item \textbf{Cross-vertical scenarios.}
        Design scenarios where decisions depend on combining
        information from multiple verticals, and study how well the
        proposed mapping and composition mechanisms behave.
\end{itemize}

% ------------------------------------------------------------
\subsection{Tooling and Automation}
% ------------------------------------------------------------

Finally, there is a practical tooling agenda.

\begin{itemize}
  \item \textbf{Vertical scaffolding.}
        Automate the creation of vertical scaffolds from compact
        \texttt{about.yaml} specifications, including directory
        layouts, adapter stubs, CEE helper modules, and example tests.
  \item \textbf{Schema- and vocab-aware IDE support.}
        Develop editor and CI tooling that keeps schemas, adapters,
        explanations, and vertical metadata in sync.
  \item \textbf{Visualization.}
        Create visualization tools that expose the geometric and
        categorical structure: civic graphs, explanation layers, and
        maps between verticals.
  \item \textbf{Reference implementations.}
        Maintain a small but representative library of open verticals
        that act as regression suites for both the theory and the
        tooling.
\end{itemize}

Together, these threads describe how the initial geometric and
bicategorical picture can grow into a stable, shared foundation for
civic data and civic explanations.

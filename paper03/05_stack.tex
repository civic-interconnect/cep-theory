% !TeX root = 00P3_cee_verticals.tex

\section{The Stack of Vertical Domains}
\label{sec:stack}

Although vertical domains can be developed independently, they share a
common architectural pattern.
We describe this pattern as a \emph{stack} layered over the CEP base.

% ------------------------------------------------------------
\subsection{Layers of the Stack}
% ------------------------------------------------------------

Conceptually, the stack has four layers:
\begin{enumerate}[label=(\arabic*),nosep]
  \item \textbf{Canonical base (CEP).}
        Schemas, vocabularies, identity,
        canonicalization, and graph normalization.
  \item \textbf{Adapters.}
        Deterministic mappings from raw sources into
        CEP entities and relationships.
  \item \textbf{Explanations (CEE).}
        Explanation types, evidence sets,
        attribution sets, and narratives over canonical graphs.
  \item \textbf{Vertical domains.}
        Slices that select subsets of the
        above three layers for specific civic questions.
\end{enumerate}

The first three layers are domain-agnostic:
they apply to any civic context expressible in CEP.
The fourth layer instantiates them for
particular questions of interest.

% ------------------------------------------------------------
\subsection{Verticals as Fibers over the CEP Universe}
% ------------------------------------------------------------

Let $\mathcal{U}_{\mathrm{CEP}}$ denote the ``universe'' of CEP schemas
and vocabularies.
Each vertical $V$ chooses a subset
$\mathcal{U}_V \subseteq \mathcal{U}_{\mathrm{CEP}}$
together with adapters and explanations.

The collection of all verticals can be
arranged as a family of fibers over
$\mathcal{U}_{\mathrm{CEP}}$:
\[
  \pi : \mathsf{Vert} \to \mathcal{U}_{\mathrm{CEP}},
\]
where the fiber over a given subset of schemas is the set of verticals
that use exactly those schemas (up to specified equivalence).

We do not formalize this as a fibration in the categorical sense,
but the analogy is useful:
moving along the base corresponds to changing
which parts of CEP are in scope;
moving within a fiber corresponds to changing
explanations or adapters while keeping the same schemas.

% ------------------------------------------------------------
\subsection{Reusability Across Verticals}
% ------------------------------------------------------------

A major benefit of the stack perspective is reusability.
Because verticals share the CEP base,
several components can be reused:

\begin{itemize}
  \item entity schemas (e.g., municipalities, regions, facilities)
        may appear in multiple verticals;
  \item identity schemes (e.g., SNFEI)
        provide cross-vertical linking;
  \item vocabularies (e.g., procedure types, violation types)
        can be shared;
  \item explanation types may be reused or adapted
        across verticals.
\end{itemize}

This manifests as maps between vertical domains,
a topic we return to in Section~\ref{sec:bicat}
and Section~\ref{sec:cases}.

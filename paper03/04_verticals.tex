% !TeX root = 00P3_cee_verticals.tex

\section{Vertical Domains}
\label{sec:verticals}

Informally, practitioners already speak of ``verticals'': procurement
verticals, health verticals, education verticals, and so on.
In most software settings these labels refer to application silos or
product lines.
In the CEP+CEE setting we adopt a more structural definition.

\begin{definition}[Vertical domain]
  A \emph{vertical domain} \(V\) consists of three interlocking
  components:
  \begin{enumerate}[label=(\alph*)]
    \item A \emph{CEP scope} \(C_V\): a selected subset of CEP schemas,
          vocabularies, and identifier schemes, together with the
          induced rewriting and identity rules on the corresponding
          entity and relationship objects.
    \item An \emph{adapter scope} \(A_V\): a set of deterministic
          adapters that map one or more raw data sources into canonical
          CEP entities and relationships within \(C_V\).
    \item A \emph{CEE scope} \(E_V\): a family of explanation types,
          evidence-set schemas, and attribution-set schemas that attach
          structured explanations to entities and relationships in
          \(C_V\).
  \end{enumerate}
\end{definition}

Intuitively, \(C_V\) specifies \emph{what in the civic universe we
  are looking at}; \(A_V\) specifies \emph{how raw data enters that
  universe}; and \(E_V\) specifies \emph{why certain parts of that
  universe are highlighted as important}.
A vertical is therefore not merely an application domain but a
\emph{sustained lens} on the canonical civic graph.

\paragraph{Examples.}

\begin{itemize}
  \item A \textbf{SME-friendly procurement} vertical might include
        entities for buyers, tenders, lots, contracts, and suppliers;
        relationships describing which buyer issues which tender and
        which contract is awarded to which supplier; adapters from
        OCDS-style procurement releases into that schema; and CEE
        explanation types that answer questions such as ``Why is this
        lot SME-friendly?''.
  \item A \textbf{community asset access} vertical might include
        entities for public assets (parks, libraries, recreation
        centres), neighbourhoods, and jurisdictions; relationships
        placing assets into neighbourhoods and neighbourhoods into
        jurisdictions; adapters from city open data portals and
        population grids; and explanations that answer ``Why is this
        neighbourhood prioritized for new assets?''.
\end{itemize}

In both cases the vertical is defined by a \emph{coherent pattern of
  use} across CEP, adapters, and CEE, not by any single file or schema.

\paragraph{Design constraints.}

Vertical domains are subject to several design constraints that
distinguish them from one-off analyses:

\begin{enumerate}[label=(\roman*)]
  \item \textbf{Reusability.}  The CEP scope \(C_V\) should reuse
        existing canonical schemas and vocabularies wherever possible,
        rather than proliferating bespoke types.
  \item \textbf{Determinism.}  Adapters in \(A_V\) must respect CEP's
        canonicalization and identity rules so that repeated runs
        produce stable entity identifiers and graph structure.
  \item \textbf{Legibility.}  Explanations in \(E_V\) must satisfy
        CEE's evidence and attribution constraints, so that their
        semantics are inspectable, testable, and comparable across
        instances and time.
  \item \textbf{Locality.}  Each vertical should be cancellable or
        extensible with minimal disruption to others: one can remove
        the community asset vertical without affecting the SME
        procurement vertical, and vice versa.
\end{enumerate}

These constraints are reflected in the repository structure: each
vertical has its own \emph{about} file, examples, adapters, and tests,
but they all sit on top of the same CEP and CEE foundations.
They ensure that verticals compose cleanly into a larger civic data
ecosystem.

% !TeX root = 00P3_cee_verticals.tex
\section*{Appendix D. Diagrammatic Intuition}
\label{app:D}
\addcontentsline{toc}{section}{Appendix D. Diagrammatic Intuition}

This appendix translates the bicategorical semantics into diagrammatic
intuition using string diagrams and geometric metaphors.

\subsection*{D.1 1-Morphisms as Flows}

Entities are points; relationships are arrows connecting them.
Adapters bend and reshape arrows while preserving alignment.

\begin{center}
  \begin{tikzcd}
    X \arrow[r, "f"] \arrow[d, dashed, "N"'] & Y \arrow[d, dashed, "N"] \\
    \Canon(X) \arrow[r, "\Canon(f)"'] & \Canon(Y)
  \end{tikzcd}
\end{center}

\subsection*{D.2 Explanations as Surfaces}

A 2-morphism $E : f \Rightarrow g$ becomes a surface spanning two arrows.
Narratives correspond to labeling this surface with evidence and attribution.

\subsection*{D.3 Vertical Columns}

Each vertical domain is a translucent column through the CEP+CEE stack.
Horizontally, arrows represent intra-domain semantics.
Curved ribbons between columns represent maps of verticals.

\subsection*{D.4 The Stack Visualization}

The geometric visualization comprises:
\begin{itemize}
  \item a hexagonal foundational plane (CEP),
  \item crystalline pipelines (adapters),
  \item floating iridescent polyhedra (explanations),
  \item vertical translucent columns (domains),
  \item light ribbons between columns (functors and 2-morphisms).
\end{itemize}


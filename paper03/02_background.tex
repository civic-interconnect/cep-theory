% !TeX root = 00P3_cee_verticals.tex
\section{Background}
\label{sec:background}

This section summarizes the ingredients on which our framework rests:
the Civic Exchange Protocol (CEP), the Contextual Evidence and
Explanations (CEE) layer, and the categorical tools we draw upon.

\subsection{The Civic Exchange Protocol (CEP)}

CEP is a schema- and rewriting-based specification for civic data.
At a high level, CEP defines:
\begin{itemize}[nosep]
  \item \emph{Entity schemas} for units such as municipalities, facilities,
        tenders, lots, contracts, programs, shelters, and so on;
  \item \emph{Relationship schemas} for links such as buyer--issues--tender,
        asset--located-in--area, facility--has--permit;
  \item \emph{Exchange and record-envelope schemas} that bundle entities
        and relationships with provenance and validation artifacts;
  \item \emph{Vocabularies} for shared codes (procedure types, violation
        types, status codes, etc.);
  \item \emph{Canonicalization and identity rules} that normalize data and
        compute stable fingerprints such as SNFEI.
\end{itemize}

The operational specification and reference implementation are maintained
in the open-source \emph{Civic Interconnect} project~\citep{cep2025spec},
which also defines concrete schemas, vocabularies, and adapters for
multiple domains (campaign finance, municipalities, environment,
education, and procurement).
We refer the reader to the companion CEP theory paper for the underlying
theory of rewriting, canonical forms, graph normalization, and identity
fingerprints, which builds on standard accounts of term rewriting
systems~\citep{baader1998term,terese2003rewriting}.

\subsection{Contextual Evidence and Explanations (CEE)}

CEE is a nascent but structurally constrained layer that sits above CEP.
Conceptually, CEE introduces three core artifacts:
\begin{description}
  \item[Evidence sets] summarizing the facts, metrics, and features that
        support a particular decision or classification.
  \item[Attribution sets] describing which models, rules, pipelines, and
        agents are responsible for that decision.
  \item[Explanation bundles] packaging evidence, attribution, and
        narrative fields for a specific subject entity or graph.
\end{description}

An explanation bundle is always anchored to one or more CEP entities or
relationships, and to an explicit \emph{explanation type}, such as:
\begin{itemize}[nosep]
  \item SME-friendly procurement lot;
  \item Priority neighborhood for community asset investment;
  \item Elevated environmental compliance risk facility;
  \item High-value education program for a given learner profile;
  \item Critical shelter or route in a resilience plan.
\end{itemize}
In this way, CEE does not operate over arbitrary raw data, but over
canonicalized CEP graphs.
This dependency is central to the semantic structure developed later:
explanations are layered \emph{on top of} canonical entities and
relationships, and they inherit their invariants from the underlying
canonicalization and identity rules.

\subsection{Category Theory and Categorical Semantics}

At a technical level, we appeal to standard notions from category theory
and bicategory theory.
Good general references include Mac~Lane's classic treatment of
categories~\citep{maclane1998categories},
modern introductions by Awodey~\citep{awodey2010category},
Leinster~\citep{leinster2014basic},
and Riehl~\citep{riehl2017category},
and applied accounts such as
Spivak's work on functorial data modeling and migration
\citep{spivak2013functorial,spivak2014category}
and Fong and Spivak's ``Seven Sketches''~\citep{fong2019seven}.
We assume familiarity with:
\begin{itemize}[nosep]
  \item categories, functors, and natural transformations;
  \item monoidal structure and composition of morphisms;
  \item bicategories, 2-morphisms, and coherence conditions.
\end{itemize}
Appendix~\ref{app:A} provides a brief, self-contained
review of the categorical notions we rely on, and Appendix~\ref{app:E}
offers a glossary-oriented view for non-specialists.

Informally, we will use the following guiding intuition:
\begin{itemize}
  \item CEP canonical graphs form (at least) a category: objects are
        canonical entities or graphs; morphisms are relationships and
        validated transformations between them.
  \item CEE explanations behave like 2-morphisms: they ``sit between''
        morphisms, refining or annotating one morphism with respect to
        another.
\end{itemize}
This is enough to motivate a bicategorical semantics without committing
to a fully formalized construction in this paper.
The remainder of the paper builds on this background to develop the
notion of vertical domains and their bicategorical structure.

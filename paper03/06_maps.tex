% !TeX root = 00P3_cee_verticals.tex
\section{Maps Between Vertical Domains}
\label{sec:maps}

Once vertical domains are regarded as structured slices of a common
bicategorical space, it becomes natural to ask how they relate to one another.
In this section we sketch three classes of ``maps between verticals'',
each corresponding to a different way of transporting structure.

\subsection{Entity-Alignment Functors}

The most basic maps are functors that align entities across verticals.
%
For instance, the same municipality may appear in both the community
asset vertical and the environmental compliance vertical.
%
Both verticals use the same CEP entity schema and the same SNFEI
identifier, but they emphasise different relationships and
explanations.

An \emph{entity-alignment functor} between verticals \(V\) and \(W\)
is a functor \(F : \mathcal{C}_V \to \mathcal{C}_W\) between the
underlying entity-relationship categories such that:

\begin{itemize}
  \item on objects, \(F\) identifies shared entities (e.g.\ a
        municipality in \(V\) with the same municipality in \(W\));
  \item on morphisms, \(F\) preserves those relationships that are
        meaningful in both verticals (e.g.\ jurisdictional embeddings).
\end{itemize}

These functors formalize the intuition that verticals live in the
same universe; they differ in focus, not in ontological commitment.

\subsection{Evidence-Transport Maps}

A richer kind of map transports \emph{evidence semantics} from one
vertical to another.
%
For example, a deprivation index used in the community asset vertical
may also be relevant in a health outcomes vertical, or risk metrics
computed in an environmental compliance vertical may inform a
resilience vertical.

Given verticals \(V\) and \(W\), an \emph{evidence-transport map}
associates to each explanation type in \(V\) a compatible explanation
type or evidence schema in \(W\), together with transformation rules
for metric names, scales, and interpretations.
%
At the bicategorical level, such a map can be viewed as a 2-functor
that acts on explanation 2-morphisms.

\subsection{Explanation-Composition Maps}

Finally, explanations from different verticals can sometimes be
composed to yield higher-level narratives.
%
For instance:

\begin{itemize}
  \item An SME-friendly procurement explanation may be composed with a
        community asset explanation to yield an account of why a
        particular procurement decision supports community economic
        resilience.
  \item An environmental risk explanation may be composed with a
        shelter-criticality explanation to express how facility risk
        impacts evacuation planning.
\end{itemize}

We refer to such constructions as \emph{explanation-composition
  maps}.
%
Formally, they appear as higher-order 2-morphisms that combine
explanations from different verticals along shared objects and
relationships.

These maps are central to the idea of a civic \emph{stack}: they show
how verticals remain modular while still participating in a coherent
global semantics.

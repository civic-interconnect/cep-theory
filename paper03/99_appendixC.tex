% !TeX root = 00P3_cee_verticals.tex
\section*{Appendix C. Proof Sketches}
\label{app:C}
\addcontentsline{toc}{section}{Appendix C. Proof Sketches}

This appendix outlines informal sketches of several propositions stated
in the main text.
Full formalization is left for future work.

\subsection*{C.1 SNFEI Uniqueness up to Canonical Equivalence}

\textbf{Claim.}
If two entities normalize to the same canonical string, then they receive
the same SNFEI, and any allowed CEP morphism preserves SNFEI.

\emph{Sketch.}
Normalization is a strict monoidal functor from raw records to canonical
strings.
SHA-256 is a deterministic function.
Thus SNFEI is invariant under any morphism commuting with normalization.

\subsection*{C.2 Explanation Bundles Form 2-Morphisms}

\textbf{Claim.}
CEE explanations satisfy the axioms of 2-morphisms in a bicategory.

\emph{Sketch.}
Given relationships $f,g : X \to Y$,
an explanation bundle assigns contextual evidence relating them.
Composition of explanations corresponds to sequential evidence refinement.
Identity explanations come from trivial evidence sets.
Associativity holds up to narrative equivalence.

\subsection*{C.3 Vertical Domains as Fibers}

\textbf{Claim.}
Each vertical domain forms a fiber over the CEP base category.

\emph{Sketch.}
Fix an entity type $E$.
All CEE explanations whose subject is $E$ form a fiber above $E$.
Domain boundaries merely restrict fibers to selected schemas and
explanation types.


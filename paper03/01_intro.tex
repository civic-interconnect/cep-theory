% !TeX root = 00P3_cee_verticals.tex

\section{Introduction}
\label{sec:intro}

The development of the Civic Exchange Protocol (CEP) was motivated by
a familiar problem in civic data work: heterogeneous sources,
idiosyncratic schemas, and ad hoc integrations make it difficult to
reason about civic processes in a principled way.

CEP responds by providing a uniform, deterministic substrate for
civic entities, relationships, exchanges, and their provenance~\cite{case2025cep}.
The entity model partitions civic data into six disjoint kinds---Actors (A),
Sites (S), Instruments (I), Events (E), Jurisdictions (J), and
Observations (O)---as defined by the Civic Accountable Entities (CAE)
ontology~\cite{case2025cae}.
CAE establishes \emph{what kinds of things exist} in civic systems;
CEP establishes \emph{how they are represented and exchanged}.
At this level the primary concerns are canonicalization, identity,
and graph-level normalization.

Contextual Evidence and Explanations (CEE) extend this substrate in a
different direction.
Where CAE defines \emph{what} civic entities are, and CEP defines
\emph{how} they flow, CEE addresses \emph{why} particular nodes,
edges, or subgraphs matter in a given context.
CEE expresses evidence sets, attribution chains, and explanation
bundles that make model behavior, rule-based systems, and analytic
pipelines \emph{legible} to both humans and machines.

Taken together, CAE, CEP, and CEE invite a higher-order point of view.
Instead of treating each use case---SME-friendly procurement,
community asset access, environmental compliance, education access,
or disaster resilience---as a separate system, we view each as a
\emph{vertical domain}: a structured slice through the same canonical
civic universe.
Each vertical selects a subset of CAE entity kinds, a family of
CEP adapters from raw data into canonical form, and a set of CEE
explanation types that capture domain-specific notions of
accountability and impact.

Once this perspective is made explicit, a striking simplification occurs.
Apparent complexity collapses into a small number of recurring patterns.
Different verticals reuse the same canonicalization machinery, the
same identity semantics, and the same explanation contracts.
They differ mainly in which portions of the civic landscape they
highlight and which explanations they attach.

The central claim of this paper is that this simplification reflects
deeper mathematical structure.
We argue that the combined CAE+CEP+CEE architecture admits a natural
\emph{bicategorical} interpretation:
\begin{itemize}
  \item \textbf{0-cells} (objects) correspond to canonical entities
        classified by CAE kinds,
  \item \textbf{1-morphisms} correspond to relationships and exchanges
        governed by CEP semantics, and
  \item \textbf{2-morphisms} correspond to CEE explanations that relate
        one interpretation of a relationship to another.
\end{itemize}

Vertical domains then appear as structured slices or fibres in this
bicategorical space, and maps between verticals correspond to functors
and natural transformations that transport both data and explanations.

This paper makes three contributions:
\begin{enumerate}
  \item We formalize \emph{vertical domains} as fibered structures over
        the CEP base category, showing how domain-specific semantics
        layer cleanly atop the shared canonical substrate.

  \item We develop a \emph{bicategorical semantics} for CEE explanations,
        capturing how evidence chains, attributions, and narratives
        compose across accountability relationships.

  \item We demonstrate the framework on two concrete verticals (SME-friendly
        procurement and community asset access), showing that diverse
        policy domains instantiate the same structural patterns.
\end{enumerate}

The remainder of the paper is organized as follows.
Section~\ref{sec:background} reviews the Civic Accountable Entities (CAE)
ontology and the categorical semantics of the Civic Exchange Protocol (CEP)
on which Contextual Evidence and Explanations (CEE) is built.
Section~\ref{sec:cep-recap} summarizes the categorical structure of CEP as the
canonical exchange layer.
Section~\ref{sec:verticals} introduces vertical domains at an operational
level, emphasizing how domain-specific entities, relationships, and
exchanges are composed.
Section~\ref{sec:stack} describes the layered stack architecture relating
CAE, CEP, jurisdictional adapters, and CEE.
Section~\ref{sec:bicat} develops the bicategorical semantics underlying
vertical composition and interaction.
Section~\ref{sec:maps} explores functorial and oplax maps between vertical
domains.
Section~\ref{sec:cases} instantiates the framework on concrete verticals,
including SME-friendly public procurement and community asset access.
Section~\ref{sec:related} discusses related work.
Section~\ref{sec:agenda} outlines a forward-looking research agenda.
Section~\ref{sec:conclusion} concludes.

In addition, the appendices collect background material and supporting
exposition intended to improve accessibility without interrupting the main
narrative.
Appendix~A reviews category-theoretic background required for the paper.
Appendix~B presents worked examples illustrating vertical composition and
explanation generation.
Appendix~C provides proof sketches for key semantic claims, including
bicategorical coherence conditions.
Appendix~D offers diagrammatic intuition using commutative and bicategorical
diagrams.
Appendix~E introduces categorical concepts specific to CEE, including
observations, explanations, and attribution.
Appendix~F provides additional plain-language explanations of categorical
terminology used throughout the paper for non-specialist readers.

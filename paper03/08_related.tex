% !TeX root = 00P3_cee_verticals.tex


\section{Related Work}
\label{sec:related}

We briefly situate this work within three strands of literature:
interoperability specifications, explainable AI, and categorical
semantics.

\subsection{Interoperability Specifications}

CEP is inspired in part by standards such as OCDS for procurement,
FHIR for health data, and PROV for provenance.
These specifications establish domain-specific schemas and sometimes limited notions of
provenance and interpretation.
CEP extends this tradition with a rewriting-based canonicalization layer and unified identity semantics.

The notion of vertical domains resonates with how profiles and
extensions are used in these communities but adds an explicit semantic
stack and categorical framing.

\subsection{Explainable AI and Model Governance}

CEE is aligned with work on model cards, data statements, and regulatory
requirements for explanation in domains such as credit, employment, and
public benefits.
However, most existing approaches treat explanations
as annotations of model artifacts, not as 2-morphisms over canonical civic graphs.
Our approach suggests a more structural integration of explanations with data semantics.

\subsection{Categorical Semantics and Applied Category Theory}

Finally, this work connects to applied category theory in databases,
open games, probabilistic programming, and causal inference.
The idea of treating explanations as 2-morphisms is reminiscent of compositional
approaches to lenses, rewrites, and double categories.
We view this paper as an invitation to bring similar tools to civic data and
governance contexts.

% !TeX root = 00P3_cee_verticals.tex

% ------------------------------------------------------------
\section{Limitations}
\label{sec:limitations}
% ------------------------------------------------------------

This paper develops a structural and categorical perspective on civic
explanations and vertical domains.
Several limitations should be made explicit.

First, the bicategorical semantics proposed here is intentionally
\emph{schematic}.
While the objects, 1-morphisms, and 2-morphisms are well motivated by
CEP and CEE practice, we do not provide a fully formalized construction
of the bicategory $\mathbf{Civ}$, nor do we prove full coherence or
universality results.
The aim is conceptual clarity rather than maximal formal generality.

Second, the scope of this work is restricted to \emph{canonical,
  revision-based civic records}.
Streaming data, probabilistic estimates, continuously updated signals,
and purely statistical aggregates are not modeled.
Extending the framework to such settings may require enriched,
temporal, or probabilistic categorical structures.

Third, explanations in CEE are treated as structured, inspectable
artifacts, not as psychological or normative guarantees.
This paper does not claim that explanations defined in this framework
are sufficient for fairness, accountability, or regulatory compliance
in all settings; rather, it provides a semantic substrate on which such
claims can be evaluated.

Finally, while vertical domains are motivated by real civic use cases,
the examples in this paper are illustrative rather than exhaustive.
Production deployments raise additional concerns around governance,
institutional incentives, data quality, and long-term maintenance that
lie beyond the present treatment.

% ------------------------------------------------------------
\section{Future Work}
\label{sec:future-work}
% ------------------------------------------------------------

The framework developed here opens several natural directions for
extension.

On the theoretical side, future work includes completing the formal
bicategorical construction sketched in Section~\ref{sec:bicat},
establishing coherence laws for explanation composition, and clarifying
the precise categorical status of vertical domains as indexed or
fibered substructures over the CEP base.

On the semantic side, richer explanation types may be developed,
including counterfactual explanations, contrastive explanations, and
policy-sensitive explanations that depend explicitly on normative or
regulatory regimes.
Integrating such explanations while preserving modularity and
inspectability remains an open challenge.

On the systems side, further work is needed to connect the semantics to
robust tooling.
This includes automated validation of explanation bundles, regression
testing of verticals across data updates, and visualization techniques
that make the layered structure of entities, relationships, and
explanations legible to practitioners.

Finally, the interaction between CEP+CEE and external governance
frameworks—such as procurement law, environmental regulation, and AI
oversight regimes—deserves sustained study.
The categorical perspective developed here provides a promising
foundation for such work, but its institutional implications remain to
be explored.

% ------------------------------------------------------------
\section{Conclusion}
\label{sec:conclusion}
% ------------------------------------------------------------

This paper has proposed a way to see civic data and civic explanations as
parts of a single geometric and categorical architecture.
The Civic Exchange Protocol (CEP) supplies a canonical substrate for
entities, relationships, identities, and provenance.
The Contextual Evidence and Explanations (CEE) layer adds contracts for
evidence, attribution, and narratives.
Vertical domains then appear as structured slices through this shared
universe: they select particular schemas, adapters, and explanation
types to answer concrete civic questions.

By interpreting entities as objects, relationships (with provenance) as
1-morphisms, and explanations as 2-morphisms in a bicategory, we obtain
a language for reasoning both \emph{within} and \emph{across} domains.
A single municipality, facility, or program can participate in multiple
verticals without duplication; explanation patterns can be transported,
compared, and composed; and new verticals can be added without
destabilizing existing ones.
In this view, civic explanations are not merely annotations on a graph
but first-class morphisms between morphisms: disciplined ways of saying
\emph{why} a relationship holds, under what assumptions, and how it
relates to other relationships.

From a practical standpoint, the framework offers a disciplined path for
building end-to-end pipelines that are simultaneously interoperable,
explainable, and mathematically coherent.
It suggests how civic technologists, regulators, and communities might
coordinate around shared abstractions—schemas, adapters, and
explanation types—rather than bespoke one-off systems.

From a theoretical standpoint, it opens a research program at the
intersection of category theory, graph rewriting, and AI transparency:
formalizing the bicategory of civic entities and explanations,
characterizing verticals via universal properties, and studying when and
how explanations can be transported between domains.

Perhaps the most striking feature is the simplification effect.
What begins as an apparently intractable landscape of heterogeneous
systems resolves, under the right abstractions, into a small collection
of reusable patterns and maps between them.
This is often the signature of a durable theory: once the correct level
of description is found, what was impossibly complicated can suddenly
appear almost self-evident.

The work ahead—including refining the formal semantics, validating
verticals on real data, and stress-testing explanation transport across
domains—is substantial but tractable.
More importantly, it is grounded in concrete civic questions:
who is affected, what decisions are being made, and how those decisions
can be explained and improved.
In that sense, the stack of vertical domains is not simply an abstract
construction; it is a proposed blueprint for civic infrastructure that
can reason about itself in ways that are legible and justifiable to the
communities it serves.

\section*{Acknowledgements}

Portions of this work were developed through human-computer collaboration
using modern computational tools.
Generative language models were used to assist with editing, formatting,
and consistency checking during manuscript preparation.
All conceptual framing, formal development, results, interpretations,
and conclusions are the author's own.
All generated suggestions were critically reviewed and validated, and
the author takes full responsibility for the content of this work.

% !TeX root = 00P3_cee_verticals.tex
\clearpage
\section*{Appendix E. Categorical Concepts for CEE}
\label{app:E}
\addcontentsline{toc}{section}{Appendix E. Categorical Concepts for CEE}

This appendix extends the categorical glossary in CEP Appendix~E to cover
concepts specific to the bicategorical semantics of CEE.
For foundational definitions of \emph{category}, \emph{functor},
\emph{natural transformation}, \emph{monoidal category}, \emph{oplax functor},
\emph{pullback}, and \emph{fibered category}, see CEP Appendix~E.

% ------------------------------------------------------------
\subsection*{E.1 Bicategory (New in CEE)}
% ------------------------------------------------------------

Formally, CEE is modeled as a bicategory whose hom-categories
capture explanatory variation over CEP morphisms.
A \emph{bicategory} generalizes a category by allowing transformations
between morphisms, called \emph{2-morphisms}.

\begin{itemize}
  \item \textbf{0-cells}: objects (civic entities)
  \item \textbf{1-morphisms}: relationships between entities
  \item \textbf{2-morphisms}: transformations between relationships
\end{itemize}

In CEE, explanations live as 2-morphisms: they relate one interpretation
of a relationship to another, providing the ``why'' behind civic claims.
This is the key structural innovation that distinguishes CEE from CEP.

% ------------------------------------------------------------
\subsection*{E.2 Fibered Category (Extended)}
% ------------------------------------------------------------

CEP Appendix~E introduces fibered categories as ``contextual layers.''
CEE extends this with multiple fibered layers over the CEP base:

\[
  p_{\mathrm{CTag}} : \mathrm{CTag} \to \mathrm{CEP}
  \qquad
  p_V : V \to \mathrm{CEP}
  \qquad
  p_{\mathrm{CEE}} : \mathrm{CEE} \to \mathrm{CEP}
\]

Each projection sends annotations, vertical semantics, or explanations
to the CEP record they describe. The fiber over a record $r$ is the
collection of all structures ``about'' $r$. This layered architecture
allows CEE to extend CEP without modifying the base protocol.

% ------------------------------------------------------------
\subsection*{E.3 Universal Property (Clarified)}
% ------------------------------------------------------------

CEP Appendix~E describes universal properties as ``uniqueness by optimality.''
In CEE, we clarify the role of SNFEI within this construction:

\begin{itemize}
  \item The \emph{stable verifiable ID} is the universal construction.
  \item SNFEI is the \emph{canonical fallback} when no external registry
        identifier (LEI, SAM UEI, etc.) is available.
  \item Any deterministic identification scheme respecting CEP invariants
        factors uniquely through the verifiable ID scheme.
\end{itemize}

% ------------------------------------------------------------
\subsection*{E.4 Summary Table}
% ------------------------------------------------------------

\begin{center}
  \small
  \begin{tabular}{p{0.24\linewidth} p{0.34\linewidth} p{0.32\linewidth}}
    \toprule
    \textbf{Concept}       & \textbf{Intuition}          & \textbf{CEE Role}               \\
    \midrule
    Category               & Things, allowed changes     & CEP record layer (see CEP)      \\
    Functor                & Structure-preserving map    & Adapters (see CEP)              \\
    Natural transformation & Coherent comparison         & Attestations (see CEP)          \\
    Monoidal category      & Combine objects             & Canonicalization (see CEP)      \\
    \textbf{Bicategory}    & Morphisms between morphisms & \textbf{CEE explanations}       \\
    Oplax functor          & Weak preservation           & Jurisdiction adapters (see CEP) \\
    Pullback               & Consistent join             & Data fusion (see CEP)           \\
    Fibered category       & Contextual layers           & CTags, verticals, CEE layers    \\
    Universal property     & Optimal construction        & Verifiable ID (SNFEI fallback)  \\
    \bottomrule
  \end{tabular}
\end{center}

% ------------------------------------------------------------
\subsection*{E.5 Closing Note}
% ------------------------------------------------------------

CEE's bicategorical structure provides the vocabulary for civic explanations:
evidence chains, attribution, and narratives become first-class mathematical
objects. Combined with CEP's categorical foundation, this enables explanations
that are coherent, auditable, and interoperable across domains.

For additional categorical terminology used in CEE (Cartesianness,
compositionality, lifting, quotienting, etc.), see Appendix~F.

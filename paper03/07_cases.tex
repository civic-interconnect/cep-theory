% !TeX root = 00P3_cee_verticals.tex
\section{Illustrative Case Studies of Vertical Domains}
\label{sec:cases}

We briefly illustrate how the preceding ideas manifest in concrete
verticals.
Each case study is treated schematically; detailed schemas
and examples are available in the accompanying implementation.

\subsection{SME-Friendly Procurement}

The SME procurement vertical $V_{\mathrm{SME}}$ selects:
\begin{itemize}[nosep]
  \item entities: buyers, tenders, lots, contracts, suppliers;
  \item relationships: buyer--issues--tender, tender--has--lot,
        contract--awarded-to--supplier, and so on;
  \item adapters: mappings from OCDS-style releases to CEP entities;
  \item explanation types: SME-friendly procurement explanations
        attached to lot entities.
\end{itemize}
The primary question is:
\begin{quote}
  Why is this lot considered SME-friendly?
\end{quote}
Evidence includes lot value relative to a threshold, procedure type, lot
structure, and possibly historical SME participation.
Attribution identifies the rule or model version and responsible agents.

In bicategorical terms, SME-friendly explanations are 2-morphisms in
$\mathbf{Civ}_{\mathrm{SME}}$ that refine classifications of lots.
They enable functorial comparison across jurisdictions and time.

\subsection{Community Asset Access}

The community asset vertical $V_{\mathrm{Assets}}$ selects:
\begin{itemize}[nosep]
  \item entities: assets (parks, libraries), areas, jurisdictions;
  \item relationships: asset--located-in--area, area--part-of--jurisdiction;
  \item adapters: from open data portals and population grids;
  \item explanation types: area access priority explanations.
\end{itemize}
The primary question is:
\begin{quote}
  Why is this neighborhood prioritized for investment in community assets?
\end{quote}
Evidence aggregates population, distance to nearest assets, asset counts
within thresholds, and equity indices.
These explanations act as 2-morphisms over morphisms 
that connect areas to assets and jurisdictions.

\subsection{Environmental Compliance, Education Access, and Resilience}

Similarly structured verticals arise for environmental compliance,
education access and value, and disaster resilience.
In each case:
\begin{itemize}
  \item a subset of CEP schemas defines the entities and relationships;
  \item adapters bring public data into canonical form;
  \item explanations attach to highlighted entities or relationships,
        answering a focused ``why'' question.
\end{itemize}
Across these verticals, common entities (e.g., municipalities) and
indices (e.g., deprivation measures) appear repeatedly, enabling maps
between sub-bicategories and composition of explanations.

These case studies illustrate how vertical domains instantiate the
abstract framework of bicategories of explanations, providing
structured, interpretable insights tailored to specific civic questions.


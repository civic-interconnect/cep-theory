% !TeX root = 00P3_cee_verticals.tex
\section*{Appendix A. Category-Theoretic Background}
\label{app:A}
\addcontentsline{toc}{section}{Appendix A. Category-Theoretic Background}

This appendix summarizes the categorical notions used in the bicategorical
semantics of vertical domains and civic explanations.
It parallels Appendix~A in the CEP semantics paper but extends it to the
2-categorical setting required for CEE.

\subsection*{A.1 Categories}

A \emph{category} $\mathbf{C}$ consists of:
\begin{itemize}
  \item a collection of \emph{objects},
  \item a collection of \emph{morphisms} (arrows) between objects,
  \item a composition law for morphisms,
  \item identity morphisms for every object,
\end{itemize}
satisfying associativity and identity axioms.

In the combined CEP+CEE setting:
\begin{itemize}
  \item objects correspond to canonical entity states,
  \item morphisms correspond to relationships and provenance-preserving
        transformations between entities.
\end{itemize}

\subsection*{A.2 Functors}

A \emph{functor} $F : \mathbf{C} \to \mathbf{D}$ maps:
\begin{itemize}
  \item each object $X$ to an object $F(X)$,
  \item each morphism $f : X \to Y$ to a morphism $F(f) : F(X) \to F(Y)$,
\end{itemize}
preserving identity and composition.

Adapters in CEP behave functorially under normalization and provenance.

\subsection*{A.3 Natural Transformations}

Given two functors $F, G : \mathbf{C} \to \mathbf{D}$,
a \emph{natural transformation} $\eta : F \Rightarrow G$ assigns to each object
$X$ a morphism $\eta_X : F(X) \to G(X)$ such that the usual naturality
square commutes.

In our setting:
\begin{itemize}
  \item natural transformations model attestations,
  \item they ensure that processing and validating commute coherently.
\end{itemize}

\subsection*{A.4 Monoidal Categories}

A \emph{monoidal category} adds a binary tensor product $\otimes$ and a unit
object $I$ satisfying coherence identities.

Canonicalization in CEP is monoidal:
it combines subcomponents into canonical strings in a deterministic order.

\subsection*{A.5 Bicategories}

A \emph{bicategory} is a weakening of a strict 2-category.
It has:
\begin{itemize}
  \item objects,
  \item 1-morphisms between objects,
  \item 2-morphisms between 1-morphisms.
\end{itemize}

Composition is associative only up to coherent isomorphism.

CEE explanations naturally form 2-morphisms in a bicategory whose
1-morphisms are provenance-preserving relationships.

\subsection*{A.6 Fibrations and Fibers}

A \emph{fibration} describes a category whose objects carry context-sensitive
structures above a base.
CEE's CTags and explanation types live in fibers above canonical entities.

\subsection*{A.7 Universal Properties}

SNFEI exhibits a universal property:
its canonical string is terminal among structures satisfying invariance,
determinism, and rewrite stability.

Understanding verticals often hinges on universal constructions across domains.


% !TeX root = 00P3_cee_verticals.tex

\section{A Bicategorical Semantics of Civic Explanations}
\label{sec:bicat}

We now sketch a bicategorical semantics that integrates CEP and CEE.
The goal is not a fully formal construction, but a coherent picture that
makes vertical domains and their interconnections mathematically
transparent.

% ------------------------------------------------------------
\subsection{CEP as a Category of Canonical Civic Graphs}
% ------------------------------------------------------------

Let $\mathbf{CEP}$ be the category whose:
\begin{itemize}[nosep]
  \item objects are canonical CEP graphs
        (entities, relationships, envelopes, identity, and provenance)
        modulo graph normalization;
  \item morphisms are well-typed graph homomorphisms and
        schema-respecting transformations,
        composed by ordinary function composition.
\end{itemize}

Adapters are then functors
$F : \mathcal{C} \to \mathbf{CEP}$
from source categories $\mathcal{C}$ of raw records.

For a vertical $V$, we restrict to a subcategory
$\mathbf{CEP}_V \subseteq \mathbf{CEP}$
as in Section~\ref{sec:verticals}.

% ------------------------------------------------------------
\subsection{CEE Explanations as 2-Morphisms}
% ------------------------------------------------------------

Explanations in CEE relate not only objects but \emph{morphisms}.
Informally:
\begin{itemize}
  \item A relationship, classification, or decision is represented by a
        morphism $f : X \to Y$ in $\mathbf{CEP}_V$.
  \item An explanation bundle describes why $f$ holds,
        which evidence it depends on,
        and which agents or models are responsible.
\end{itemize}

Given two morphisms $f,g : X \to Y$
(for example, two different ways of classifying the same lot or facility),
an explanation can be understood as a 2-morphism
\[
  \alpha : f \Rightarrow g
\]
witnessing how $f$ is obtained from $g$
under a particular explanatory contract,
or how $f$ is justified relative to a baseline.

This motivates the following heuristic construction.

\begin{conjecture}
  There exists a bicategory $\mathbf{Civ}$ whose:
  \begin{itemize}
    \item objects are canonical civic graphs
          (or families of entities);
    \item 1-morphisms are CEP relationships and graph transformations
          with provenance;
    \item 2-morphisms are CEE explanations relating 1-morphisms
          under explanation types and contracts.
  \end{itemize}
\end{conjecture}

We do not claim to have constructed $\mathbf{Civ}$ in full generality,
but the vertical domains we study can be understood as fragments of such
a bicategory.

% ------------------------------------------------------------
\subsection{Verticals as Sub-bicategories}
% ------------------------------------------------------------

For each vertical $V$, we may consider the full sub-bicategory
$\mathbf{Civ}_V$ of $\mathbf{Civ}$ whose:
\begin{itemize}[nosep]
  \item objects are those graphs built from $\mathcal{E}_V$
        and $\mathcal{R}_V$;
  \item 1-morphisms are CEP morphisms in $\mathbf{CEP}_V$;
  \item 2-morphisms are explanations in $\Xi_V$
        attaching to those morphisms.
\end{itemize}

This aligns with the architectural view of Section~\ref{sec:stack}:
each vertical is a slice of the full bicategory,
with its own adapters and explanation types.

% ------------------------------------------------------------
\subsection{Maps Between Vertical Domains}
% ------------------------------------------------------------

Once we acknowledge this bicategorical structure,
maps between verticals become semantically meaningful.
For verticals $V$ and $W$ we can study:

\begin{itemize}
  \item \emph{Entity alignment functors}
        $F : \mathbf{CEP}_V \to \mathbf{CEP}_W$
        that identify shared entity types
        (e.g., municipalities, regions, facilities).
  \item \emph{Evidence transport}
        that lifts metrics or indices from one vertical to another
        (e.g., deprivation indices used in both health
        and community asset explanations).
  \item \emph{Explanation composition}
        where 2-morphisms in $\mathbf{Civ}_V$
        and $\mathbf{Civ}_W$ compose to yield a joint explanation
        (e.g., combining SME-friendly procurement
        with local economic impact).
\end{itemize}

These constructions are sketched in more concrete form in
Section~\ref{sec:cases}.

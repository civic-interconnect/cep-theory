% !TeX root = 00P2_cep_semantics.tex

\section{Preliminaries}
\label{sec:preliminaries}

This section reviews the mathematical tools used throughout the
paper and summarizes the structural elements of the Civic Exchange
Protocol (CEP).
We assume only standard familiarity with category theory,
drawing primarily from Mac~Lane~\cite{maclane1971categories}
and Spivak~\cite{spivak2014category} for the treatment of categories
as models of data and structure-preserving transformations.

% ------------------------------------------------------------
\subsection{Category-Theoretic Background}
% ------------------------------------------------------------

We recall only the categorical notions required for the development of
CEP semantics:

\begin{itemize}
  \item \textbf{Categories}: collections of objects and morphisms equipped with associative composition and identity arrows.

  \item \textbf{Functors}: structure-preserving mappings between categories,
        sending objects to objects and morphisms to morphisms
        in a way that respects identities and composition.

  \item \textbf{Natural transformations}: morphisms between
        functors, providing coherent comparisons between system-level
        views of the same underlying data.

  \item \textbf{Monoidal categories}: categories equipped with a
        tensor product $\otimes$ and unit object $I$, allowing formal
        reasoning about concatenation, aggregation, and combination of
        structured data streams.

  \item \textbf{Oplax functors}: functors that preserve monoidal or
        structural properties up to a controlled relaxation, used here to
        model schema adapters that preserve meaning even when strict
        equivalence between jurisdictions cannot be enforced.
\end{itemize}

These constructions provide a natural language for expressing CEP's
compositional structure:
canonicalization becomes a monoidal functor,
envelopes become natural transformations, and adapters become oplax
mediators between local and global schema categories.

% ------------------------------------------------------------
\subsection{CEP Structural Recap}
% ------------------------------------------------------------

CEP defines four record families:

\begin{itemize}
  \item \textbf{Entities}: civic actors (organizations, agencies, districts, individuals).

  \item \textbf{Relationships}: directed or undirected structural
        links between entities (membership, control, affiliation, reporting lines).

  \item \textbf{Exchanges}: flows of value, information, or action
        between entities (payments, transfers, notifications).

  \item \textbf{Context tags}: optional, non-canonical annotations
        that express interpretive, analytic, or contextual facts about a record.
\end{itemize}

All record families are wrapped in a shared \emph{record envelope}
that provides:

\begin{itemize}
  \item schema and vocabulary references,
  \item revision numbers and lifecycle status,
  \item attestation metadata,
  \item timestamps describing observation and validity intervals,
  \item stable CEP identifiers (verifiable IDs).
\end{itemize}

The envelope separates canonical, identity-bearing components of a
record from contextual or analytic metadata, ensuring both stability
and extensibility.

% ------------------------------------------------------------
\subsection{The A-I-E Criterion}
% ------------------------------------------------------------

Not all civic data qualifies for inclusion in the CEP category.
We formalize the admission criterion as follows:

\begin{definition}[Admissible-Identifiable-Exchangeable (A-I-E)]
  \label{def:aie}
  A civic data record satisfies the \emph{A-I-E criterion} if and only if:
  \begin{enumerate}
    \item \textbf{Admissible (A)}: The record conforms to a declared CEP schema
          and passes validation against its \texttt{recordSchemaUri}.
          The payload instantiates one of the six CAE entity kinds (Actor, Site,
          Instrument, Event, Jurisdiction, Observation) or a valid relationship
          or exchange between such entities.

    \item \textbf{Identifiable (I)}: The record's canonical payload produces a
          stable identifier under CEP canonicalization---either a registry ID
          (LEI, SAM UEI, etc.) or a derived SNFEI when no registry ID is available.

    \item \textbf{Exchangeable (E)}: The record is wrapped in a valid envelope
          with attestation, timestamp, and revision metadata sufficient for
          provenance-preserving exchange across system boundaries.
  \end{enumerate}
\end{definition}

The A-I-E criterion acts as a \emph{filter}: it determines which records
from heterogeneous civic data sources may be lifted into $\mathbf{CEP}$.
Records failing any condition are excluded from the category until
remediated by canonicalization, schema alignment, or envelope construction.

% ------------------------------------------------------------
\subsection{Canonicalization and Identifiers}
% ------------------------------------------------------------


For entity records, CEP constructs a canonical string from
jurisdiction-normalized components such as name, address, and
formation date.
A SHA-256 hash of this canonical string yields a
stable identifier known as the \emph{Structured Non-Fungible Entity Identifier} (SNFEI).

In this paper, we treat canonicalization as a deterministic, strictly monoidal process:
concatenation of components corresponds to a tensor-like operation, and hashing
corresponds to an endofunctor that collapses equivalence classes of canonical strings
to identity objects.
This perspective will be formalized in Section~\ref{sec:canonicalization}.

The stability of SNFEIs under valid record transformations is a core invariant
that underpins CEP's interoperability guarantees.

The formal semantics developed in subsequent sections establish the precise
conditions under which this stability holds.

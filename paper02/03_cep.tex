% !TeX root = 00P2_cep_semantics.tex


\section{\texorpdfstring{The Category $\mathbf{CEP}$}{The Category CEP}}
\label{sec:cep}

This section formalizes the structural semantics of the Civic Exchange
Protocol in categorical terms.
We introduce the category
$\mathbf{CEP}$, whose objects are attested record states and whose
morphisms capture provenance-preserving transformations.
The purpose is
not to replace CEP's operational semantics, but to express its invariants
and data-integration behavior in a compositional language.

\subsection{CEP as a unified rewriting framework}

We treat CEP as a family of rewriting systems glued together by a common
category-theoretic backbone.
At the lowest level, canonicalization rewrites
raw strings into canonical forms.
At the schema level, adapters rewrite
source-specific records into CEP entities and relationships.
At the graph
level, provenance and entity-merge operations rewrite collections of records
into coherent, versioned entity graphs.
Each layer uses a stratified,
strategy-governed set of rewrite rules rather than assuming that all
transformations commute.

In this view, adding a new civic domain (e.g., campaign finance, environmental
permits, educational programs) means specifying:
(i) a domain vocabulary,
(ii) a set of rewrite rules for codes, names, and identifiers, and
(iii) adapters that connect domain-specific sources into the CEP category.
Sections \ref{sec:canonicalization} to \ref{sec:ctags} make these rewriting layers
explicit, and Section~\ref{sec:apps} outlines several domain-specific instantiations.

% ------------------------------------------------------------
\subsection{Objects: Attested Record States}
% ------------------------------------------------------------

The objects of the category $\mathbf{CEP}$ are \emph{attested record states}
conforming to CEP's schema and integrity constraints.
The category is intentionally scoped to the civic accountability domain:
only records that either instantiate, or can be mapped to,
a valid \emph{Admissible-Identifiable-Exchangeable} (A-I-E) exchange pattern
(Definition~\ref{def:aie}) are included.

Objects of $\mathbf{CEP}$ are \emph{well-typed, time-stamped record states}.
Each record consists of a payload (entity, relationship, or exchange)
together with its envelope, which includes schema references,
revision numbers, attestations, timestamps, and stable identifiers.

Formally:
\begin{equation}
  \mathrm{Ob}(\mathbf{CEP})
  = \{\, R \mid R = (\mathrm{Payload}, \mathrm{Envelope})
  \text{ is a valid CEP record admissible under the A-I-E criterion} \}.
\end{equation}

Each object $R$ carries a fixed record kind (entity, relationship, or
exchange) and is assumed to satisfy JSON Schema validity with respect to
its declared \texttt{recordSchemaUri}.

% ------------------------------------------------------------
\subsection{Morphisms: Provenance-Preserving Transformations}
% ------------------------------------------------------------

Morphisms in $\mathbf{CEP}$ represent admissible, provenance-preserving
transformations between record states.
A morphism $f : R \to R'$ models a transition such as an amendment, attestation
update, relationship creation, or audit step.
Morphisms are total: only transformations that satisfy CEP's invariants
constitute morphisms in the category.
Transformations that would violate these invariants are excluded rather
than represented within $\mathbf{CEP}$.

To qualify as a morphism, $f$ must preserve CEP's core invariants:
\begin{enumerate}
  \item \textbf{Schema validity}: 
        $\mathrm{Schema}(R) \to \mathrm{Schema}(R')$ must be respected,
        meaning $R'$ remains valid under the same or a successor schema.

  \item \textbf{Revision monotonicity}:  
        $\mathrm{Revision}(R) \leq \mathrm{Revision}(R')$.
        Updates must advance (not rewind) the revision counter.

  \item \textbf{Identity invariance}:  
        The canonical identifier (SNFEI) associated with the canonical
        payload of $R$ must match that of $R'$.
        This ensures that morphisms do not alter identity-bearing fields.
\end{enumerate}

Intuitively, morphisms encode permissible "ways a record can change"
while respecting immutability of identity and the integrity constraints
of the envelope.


% ------------------------------------------------------------
\subsection{Finite Limits and Consistent Joins}
% ------------------------------------------------------------

We assume $\mathbf{CEP}$ is \emph{finitely complete}, meaning it
admits all finite limits.
Among these, the most important for data
integration is the \emph{pullback}, which formalizes the notion of a
consistent join between heterogeneous record fragments.

Given two morphisms $f : R \to A$ and $g : R' \to A$, their pullback
is an object $P$ equipped with morphisms $p_1 : P \to R$ and
$p_2 : P \to R'$ satisfying the usual universal property.
The object $A$ may be any CEP record kind (entity, relationship, or exchange),
reflecting the fact that consistency may be required at multiple structural
levels within civic accountability data.
In practice, entity-typed pullbacks are privileged, as canonical identity
stability is strongest at the entity level.

\FigureCallout{The Categorical Pullback (Consistent Join)}{
  A pullback of $f : R \to A$ and $g : R' \to A$ is an object $P$ that
  represents the most specific record state compatible with both $R$ and
  $R'$ when they assert information about the same underlying entity or
  relationship $A$.
  For example, joining a vendor registry record $R$ with a contracting
  record $R'$ along a shared entity type $A$ yields $P$, the unique
  maximal consistent integration of both fact sets.
}

This provides a principled mechanism for data fusion, deduplication, and
consistency checking within and across jurisdictions.


% ------------------------------------------------------------
\subsection{Subobjects}
% ------------------------------------------------------------

Subobjects in $\mathbf{CEP}$ correspond to \emph{partial, typed views}
of records.
Examples include:

\begin{itemize}
  \item extracting only the attestation sequence of a record,  
  \item projecting a relationship record onto one of its participating entities,
  \item isolating the canonical payload while omitting context tags.  
\end{itemize}

Because subobjects inherit morphisms from the ambient category,
substructure analysis, such as reasoning about what remains invariant
under updates, can be expressed cleanly and compositionally.
Intuitively, this means that when part of a record is updated, any
well-defined subpart evolves in a compatible way without requiring
special-case rules.

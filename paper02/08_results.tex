% !TeX root = 00P2_cep_semantics.tex

\section{Interoperability Results}
\label{sec:results}

The categorical semantics developed in the preceding sections yield
formal guarantees for CEP's core interoperability claims.
These results follow from functoriality, naturality, monoidal structure, and the oplax
semantics of jurisdictional adapters.
All results are stated relative to the category $\mathbf{CEP}$ and
therefore apply only to admissible record states and total,
invariant-preserving transformations.

% ------------------------------------------------------------
\subsection{Functorial Consistency}
% ------------------------------------------------------------

We first establish coherence between record evolution and attestation.

\begin{lemma}[Functorial Consistency of Attested Evolutions]
  \label{lemma:functorial-consistency}
  Let $f : R \to R'$ and $g : R' \to R''$ be morphisms in
  $\mathbf{CEP}$, and let $\alpha : \mathcal{E} \Rightarrow \mathcal{E}'$
  be the natural transformation representing attestation.
  Then:
  \[
    \mathcal{E}'(g \circ f)
    \;=\;
    \alpha_{R''} \circ \mathcal{E}(g \circ f)
    \;=\;
    \mathcal{E}'(g) \circ \mathcal{E}'(f).
  \]
\end{lemma}

\begin{proof}
  Naturality of $\alpha$ gives
  \(
  \alpha_{R'} \circ \mathcal{E}(f)
  =
  \mathcal{E}'(f) \circ \alpha_R
  \)
  for all $f : R \to R'$.
  Composing these constraints and using associativity
  of morphisms in $\mathbf{CEP}$ yields the claim.
\end{proof}

This shows that CEP's update--attestation workflow behaves coherently
under composition, supporting reliable audit chains and immutability of
provenance traces.

% ------------------------------------------------------------
\subsection{Identifier Preservation}
% ------------------------------------------------------------

We now formalize the stability of canonical identifiers under all valid
record evolutions.

\begin{theorem}[Identifier Preservation]
  \label{thm:identifier-preservation}
  Let $f : R \to R'$ be any morphism in $\mathbf{CEP}$.
  Let $\mathcal{C}$
  denote the canonicalization functor and
  $H$ the deterministic hash used to compute the SNFEI.
  Then:
  \[
    H(\mathcal{C}(R)) \;=\; H(\mathcal{C}(R')).
  \]
\end{theorem}

\begin{proof}
  A valid $\mathbf{CEP}$ morphism $f$ preserves all canonical components
  of the record.
  Strict monoidality of $\mathcal{C}$ therefore implies
  $\mathcal{C}(R)=\mathcal{C}(R')$.
  Since $H$ is a pure function, the
  resulting SNFEI values coincide.
\end{proof}

This theorem provides the mathematical foundation for CEP's claim of
\emph{identity stability across revisions}.

% ------------------------------------------------------------
\subsection{Cross-Jurisdiction Reconciliation}
% ------------------------------------------------------------

Jurisdictional adapters reconcile heterogeneous local schemas with the
global CEP vocabulary.
We show that their oplax semantics preserves canonical identity across compositions.

\begin{theorem}[Canonical Equivalence Preservation Under Adapters]
  \label{thm:adapter-preservation}
  Let $\mathcal{A}_1$ and $\mathcal{A}_2$ be valid jurisdictional
  adapters (Section~\ref{sec:adapters}).
  If two local records $R$ and $R'$ satisfy
  \[
    H(\mathcal{C}(\mathcal{A}_1(R)))
    \;=\;
    H(\mathcal{C}(\mathcal{A}_1(R'))),
  \]
  then after composition with a second adapter,
  \[
    H(\mathcal{C}(\mathcal{A}_2(\mathcal{A}_1(R))))
    \;=\;
    H(\mathcal{C}(\mathcal{A}_2(\mathcal{A}_1(R')))).
  \]
\end{theorem}

\begin{proof}
  A valid adapter is oplax: it may weaken structure but cannot introduce
  new canonical content.
  Thus each $\mathcal{C}(\mathcal{A}_i(R))$ is defined and
  invariant under further valid transformations.
  Strict monoidality of $\mathcal{C}$ ensures that canonical strings are
  preserved under composition of such structure-weakening functors.
  Applying $H$ yields the result.
\end{proof}

\noindent
This theorem shows that multi-stage, cross-jurisdiction pipelines respect
CEP's canonical identity invariants.
It provides the formal backbone for
\emph{globally consistent, locally autonomous interoperability}.

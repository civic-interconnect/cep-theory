% !TeX root = 00P2_cep_semantics.tex


\section{Limitations and Future Work}
\label{sec:limitations}

The categorical core presented in this paper captures identity,
canonicalization, provenance, and interoperability for discrete,
revision-based civic records.
Several important extensions remain outside the present scope.

Many civic systems also generate data that is statistical, uncertain, or
continuously updated (e.g., longitudinal indicators, evolving aggregates).
Incorporating such information may require probabilistic semantics or
temporal indexing, which are not modeled in the current framework.

Jurisdictions evolve data models over time.
Although CEP supports adapters for structural variation,
a full account of schema evolution, including additions,
deprecations, and long-term migration paths, remains future work.

\paragraph{Multi-Stage and Nested Exchanges.}

Some workflows involve layered or multi-party processes
(e.g., multi-level budget allocations, nested reporting pipelines).
These may benefit from higher-structured categorical tools,
but such extensions lie beyond the scope of this foundational treatment.

\paragraph{Rule Sensitivity in Canonicalization.}

Certain linguistic cases (such as expansions of abbreviations like “S.A.”)
require stratified rule ordering within the normalization pipeline rather
than treating all rewrite rules as freely permutable.
This refinement does not affect the well-definedness of the canonicalization
function but does highlight the need for continued empirical tuning of rule strata.

\medskip

The present semantics establish a robust core for identity and interoperability.
Extending CEP to the richer data practices found across governments and civic ecosystems
is a key direction for future work.

% ------------------------------------------------------------
\section{Conclusion}
\label{sec:conclusion}

We presented a categorical semantics for the Civic Exchange Protocol
that unifies canonicalization, provenance, adapters, and context tags
into a coherent mathematical framework.
This perspective makes explicit the invariants that govern identity,
record evolution, and interoperability across heterogeneous civic data systems.

Canonicalization was formulated as a deterministic monoidal functor,
ensuring stable identifiers and well-defined equivalence classes.
Jurisdictional adapters were modeled as oplax functors, capturing how
local structure may be weakened while preserving global identity.
Context tags were expressed via a fibered category, isolating
interpretive annotations from canonical record content.
Together, these structures yield formal guarantees for identifier preservation,
compositional provenance, and cross-jurisdiction reconciliation.

The resulting semantics provides a rigorous foundation for validation,
verification, and future extensions of CEP.
It enables principled design of domain schemas, vocabularies,
and interoperability standards, while remaining extensible to evolving civic workflows.
As civic data ecosystems continue to grow in scale and complexity,
categorical methods offer a durable and expressive language for ensuring that shared
identities and exchanges remain consistent, transparent, and reliable.


\section*{Acknowledgements}

Portions of this work were developed through human-computer collaboration
using modern computational tools.
Generative language models were used to assist with editing, formatting,
and consistency checking during manuscript preparation.
All conceptual framing, formal development, results, interpretations, and conclusions
are the author's own.
All generated suggestions were critically reviewed and validated, and
the author takes full responsibility for the content of this work.